%% @@PACKAGE @@VERSIONB - @@DATE
%% The LaTeX package @@PACKAGE - version @@VERSION (@@DATE) build @@BUILD
%% #PACKAGE.sty
%% -------------------------------------------------------------------------------------------
%% Copyright (c) 2018 by Andreas Storvik Strauman
%% -------------------------------------------------------------------------------------------
%% This work may be distributed and/or modified under the
%% conditions of the LaTeX Project Public License, either version 1.3c
%% of this license or (at your option) any later version.
%% The latest version of this license is in
%%   http://www.latex-project.org/lppl.txt
%% and version 1.3c or later is part of all distributions of LaTeX
%% version 2008/05/04 or later.
%% This work has the LPPL maintenance status `author-maintained'.
%% This work consists of all files listed in README.txt
\documentclass{article}
\usepackage[all]{tcolorbox}
\usepackage{needspace}
\usepackage{tabularx}
\makeatletter
\lstdefinestyle{mydocumentation}{style=tcbdocumentation,
  classoffset=0,
  texcsstyle=*\color{blue},
  % LaTeX and other packages
  moretexcs={arrayrulecolor,draw,includegraphics,ifthenelse,isodd,lipsum,path,pgfkeysalso},
  classoffset=1,
  % tcolorbox macros
  moretexcs={% core
    makeset,phead,buildset,select,exclude,Trigger,At
  },
  texcsstyle=*\color{Definition}\bfseries,
  % classoffset=2,% restore default
  % texcsstyle=*\color{cyan},
  % literate=*%
  %     {\\!}{{\textcolor{cyan}{\textbackslash{}!}}}2
  %     {\\\\}{{\textcolor{cyan}{\textbackslash{}\textbackslash{}}}}2,
  classoffset=0,% restore default
  }
  % \makeatletter
  % \lst@AddToHook{SelectCharTable}
  %     {\ifx\lst@literate\@empty\else
  %          \expandafter\lst@Literate\lst@literate{}\relax\z@
  %      \fi}
  % \makeatother
\newtcolorbox{marker}[1][]{enhanced,
    before skip=2mm,after skip=3mm,
    boxrule=0.4pt,left=5mm,right=2mm,top=1mm,bottom=1mm,
    colback=yellow!50,
    colframe=yellow!20!black,
    sharp corners,rounded corners=southeast,arc is angular,arc=3mm,
    underlay={%
      \path[fill=tcbcol@back!80!black] ([yshift=3mm]interior.south east)--++(-0.4,-0.1)--++(0.1,-0.2);
      \path[draw=tcbcol@frame,shorten <=-0.05mm,shorten >=-0.05mm] ([yshift=3mm]interior.south east)--++(-0.4,-0.1)--++(0.1,-0.2);
      \path[fill=yellow!50!black,draw=none] (interior.south west) rectangle node[white]{\Huge\bfseries !} ([xshift=4mm]interior.north west);
      },
    drop fuzzy shadow,#1}

%!TEX root = main.tex
%\BDOC
%§ref
%\ownLineNoSpacesGotIt
% This is to annoy the user enough to get his attention about the requirements of the \!refEnv{problem}, \!refEnv{solution} and \!refEnv{intro} environments.\\
% Compilation wont work unless \end{problem}, \end{solution} and \end{intro} are on their own lines and without any spaces. This warning can be removed by doing \def\ownLineNoSpacesGotIt{} before \usepackage{exbank}.
%\EDOC
\makeatletter
\@ifundefined{ownLineNoSpacesGotIt}{
\@latex@warning{Compilation wont work unless \string\end{problem} and \string\end{solution} are on their own lines and without any spaces. This warning can be removed by doing \string\def\string\ownLineNoSpacesGotIt{} before \string\usepackage{exbank}}
}{}
\@ifundefined{exercisesDir}{
\gdef\exercisesDir{exercises}
}{}
\makeatother


\global\let\incl = \input
\usepackage{xstring}
\usepackage{pgffor}
\usepackage{scrextend}
\usepackage{comment}
\usepackage{calc}

%!TEX root = main.tex
% This file contains everything to do with translation logic.
\makeatletter
\pgfkeys{
 /exbanki18n/.is family, /exbanki18n,
  default/.style = {Problem = Problem, Solution = Solution},
  Problem/.estore in = \@tr@Problem,
  Solution/.estore in = \@tr@Solution
}
%:§lang
%:=\translateExBank{Translation key/vals}
%:This is to translate the text inside the package. As of now the available key/values are
%:\!begin{itemize}
%:\!item Problem
%:\!item Solution
%:\!end{itemize}
%:The Norwegian translation would then be done with
%:!\translateExBank{Problem=Oppgave, Solution=Løsning}
%:-
\newcommand{\translateExBank}[1]{
  \pgfkeys{/exbanki18n, default, #1}%
}
\translateExBank{}

%
\gdef\@tr#1{
\@ifundefined{@tr@#1}{#1}{
  \csname @tr@#1\endcsname
}
}
\makeatother

%!TEX root = main.tex
% This file contains definitions of the \At and \Trigger command

%:§ref
\makeatletter
%% Adds macro to queue
\let\ea = \expandafter
\begingroup\lccode`\|=`\\
\lowercase{\endgroup\def\removebs#1{\if#1|\else#1\fi}}
\ifcsname at@verbose\endcsname
\global\def\@triggerLog#1{\@latex@warning{--\string\TRIGGER: #1}}%
\else
\global\let\@triggerLog\@gobble%
\fi
%:This package also includes some extra stuff. For example the \At and \Trigger
%:!TOC
%:=\At{AnyMacro}
%:Here you can send any macro because it isn't evaluated! For example \At\BeginSomething is fine and even if \BeginSomething is not defined. Also and when using \Trigger it just ignores it if it didn't exist. It's pretty similar in function as to \AtBeginDocument.
%:! \At\BeginSomething{DoSomething}
%:! Which is triggered with
%:! \Trigger\BeginSomething, this evaluates to DoSomething
%:-
%% Usage: \At\BeginSomething (\BeginSomething) doesn't have to be defined
\newcommand{\At}[2]{
  \edef\mname{\exbank@macroname{#1}}%
  \ifcsname At@\mname\endcsname%
    \ea\g@addto@macro\csname At@\mname\endcsname{#2}%
  \else%
    \ea\gdef\csname At@\mname\endcsname{#2}%
    \AtEndDocument{%
        \expandafter\let\csname At@\mname\endcsname\relax%
    }
  \fi%
}
%% Triggers queue generated by \At
%% Usage: \Trigger\BeginSomething (\BeginSomething) doesn't have to be defined
%:§triggers=Triggers
%:=\Trigger{Any Macro}
%: See \refCom{At}\\
%: Available triggers:\\
%:$$triggers
%:-
\newcommand{\Trigger}[1]{%
  \edef\mname{\exbank@macroname{#1}}%
  \ea\@ifundefined{At@\mname}{}{%
  \@triggerLog{\mname}\csname At@\mname\endcsname%
  }%
}
\makeatother

\long\def\about#1{}

%!TEX root = main.tex
% This file contains \At-commands that are responsible for formatting
% exercise headers and exercise-related styles.
\makeatletter
%:§margs
%:Note that these are all \!emph{\!\textbf{lengths}}
%:and should be used as e.g.
%:!\setlength{\pMarginBelow}
%:-
%:=\pMarginBelow
%:Distance below Problem \# header
\newlength{\pMarginBelow}
%:=\pMarginAbove
%:Distance above Problem \# header
\newlength{\pMarginAbove}
%:=\pMarginLeft
%: Problem header: distance from the default left margin
\newlength{\pMarginLeft}
%:=\ppMarginBelow
%: Part problem: distance from the end of the part problem to the next item
\newlength{\ppMarginBelow}
%:=\ppMarginAbove
%: Part problem: distance above the start of the part problem to the previous item
\newlength{\ppMarginAbove}
%:=\ppMargin
%: Part problem: how far away from the text the part problem labels are.
\newlength{\ppMargin}
%:=\introMargin
%: The offset of margins for intros
\newlength{\introMargin}
\newlength{\solutionMarginLeft}
\newlength{\solutionMarginAbove}
%:-

\edef\defaultLeftMargin{\the\dimexpr1in+\hoffset}



\setlength{\pMarginLeft}{-2em}
\setlength{\ppMargin}{-0.5em}
\setlength{\solutionMarginLeft}{\ppMargin}
\setlength{\solutionMarginAbove}{1em}
\setlength{\pMarginBelow}{0em}
\setlength{\pMarginAbove}{\baselineskip}
\setlength{\introMargin}{-1.5em}
\setlength{\ppMarginBelow}{\baselineskip}
\setlength{\ppMarginAbove}{0em}


\setlength\parindent{0pt}
\newbox\ppmarkbox
\newlength\markskip
\setlength\markskip{4\baselineskip}

\setbox\ppmarkbox\hbox{%
\vrule\@height.7\markskip
        \@depth.3\markskip%
        \@width\z@}%
\def\markstrut{\relax\ifmmode\copy\ppmarkbox\else\unhcopy\ppmarkbox\fi}


\newcommand{\@atMargin}[2]{\markstrut\vadjust{\@domark{#1}{#2}}}

\newcommand{\@domark}[2]{%
    \hskip#2\relax\vbox to 0pt{%
      \kern-\dp\ppmarkbox\relax%
      \smash{\llap{#1}}%
      \vss%
    }%
% \vskip\the\dimexpr-\markskip\relax%
}

\def\solMargin{\dimexpr \ppMargin\relax}
\def\vsSize{1em}
\def\vs{\vspace{\vsSize}}
%:§count
%:\!docCounter{problemcounter}-counter holds the current problem number and
%:\!docCounter{partproblemcounter}-counter holds the current partproblem \!emph{number}.
%:-
\newcounter{problemcounter}
\setcounter{problemcounter}{0}
\newcounter{partproblemcounter}

\At\VeryBeginProblem{%
	\stepcounter{problemcounter}%
	\setcounter{partproblemcounter}{0}%
	\vspace*{\pMarginAbove}%
  \strut\vadjust{\vbox to 0pt{\if\exbank@opt@doMargins\@isTrue\hskip\pMarginLeft\fi{\exbank@opt@problemHeader}\vss}}\par%
	\vspace*{\the\dimexpr\baselineskip+\pMarginBelow}%
}
\At\EndProblem{%
\tighten@paragraph%
}
\At\VeryBeginPartproblem{%
	\stepcounter{partproblemcounter}%%
	\bgroup%
	% If we should display the meta counter, then%
	\if\@displayMetaCounter\@isTrue\relax%%
		\ex@before\exbank@opt@partProblemHeader{{\Large\themetacounter}:}%
	\fi%

\if\exbank@opt@doMargins\@isTrue\relax%
\@atMargin{{\exbank@opt@partProblemHeader}}{\ppMargin}%
\ignorespaces%
\else{\exbank@opt@partProblemHeader}\fi%
\ignorespaces}%
\At\EndPartproblem{%
	\egroup\tighten@paragraph\par%
	\vspace*{\pMarginBelow}%
}

\At\BeginSolution{\hfill\break\vspace*{\solutionMarginAbove}{\exbank@opt@solutionHeader}}
\At\EndSolution{}

%!TEX root = main.tex


\makeatletter

\global\let\DisplaySolution\DisplaySolutiontrue

\gdef\isFalse{0}
\gdef\isTrue{1}
\gdef\DisplayProblem{\isTrue}

%%%% NB: Difference between \DisplaySolution and \DisplaySolutions
\gdef\@DisplaySolutions{\isFalse}
\gdef\DisplaySolutions{\xdef\@DisplaySolutions{\isTrue}\@latex@warning{Showing solutions}}
\AtBeginDocument{
  \if\@DisplaySolutions\isFalse
    \@latex@warning{Hiding solutions. Show them with \string\DisplaySolutions}
  \fi
}
\global\let\do@ProcessCutFile = \ProcessCutFile
%% Problems
\generalcomment{problem}{
  %%%% NB: Difference between \DisplaySolution and \DisplaySolutions
  \edef\DisplaySolution{\@DisplaySolutions}
  %% BeginPartproblemHard is before regardless whether
  %% the problem is included or not.
  \Trigger\DecideProblemDisplay
  \begingroup
    \if\DisplayProblem\isFalse
      \def\ProcessCutFile{}
    \else
      \Trigger\BeginPartproblem
    \fi
    }{
    \if\DisplayProblem\isFalse\else
      \Trigger\EndPartproblem
    \fi
  \endgroup
}
%% Solutions
\generalcomment{solution}
{
\Trigger\AtBeginSolutionHard
\begingroup
  \if\@DisplaySolutions\isTrue
    \if\DisplayProblem\isFalse
      \xdef\DisplaySolution{\isFalse}
    \fi
  \fi
  % \inspw{\DisplaySolution}
  \if\DisplaySolution\isTrue
    \Trigger\BeginSolution
  \else
    \def\ProcessCutFile{}
  % \begingroup
  % \edef\tmp{\def\noexpand\CommentCutFile{answer.tex}}
  % \tmp
    % \def\ProcessCutFile{\do@ProcessCutFile}
  \fi
}{
% \input{\CommentCutFile}
\if\DisplaySolution\isTrue
\Trigger\EndSolution
\fi

\Trigger\EndSolutionHard
\endgroup
}
%% Problem introductions
\newenvironment{intro}{}{}

% This file contains the definition of the \isin macro
%:§ref
%:=\isin{haystack}{needle}{True}{False}
%: \meta{haystack} is a comma separated list of integers\\
%: \meta{needle} is an integer\\
%: Executes \meta{True} if \meta{needle} is found in \meta{haystack}
%: else executes \meta{False}
%:-
% #1: haystack
% #2: needle
% #3: action if found in list
% #4: action if not found in list
\gdef\isin#1#2#3#4{
  \def\needle{#2}
  \def\haystack{#1}
  \def\isFalse{0}
  \let\isInList = \isFalse
  \IfInteger{\haystack}{
    \if\haystack\needle
    #3
    \else
    #4
    \fi
  }{
  \foreach \pp in #1{
    \if\pp\needle
      \gdef\isInList{1}
      #3
    \fi
  }
  \if\isInList\isFalse
    #4
  \fi
  }
}

%!TEX root = main.tex
% This file contains the logic of the set making and set building.
% It also decides whether or not a problem,intro and/or solution
% should be built
%:§makingSets
%:-
\let\ea = \expandafter
\newcounter{metacounter}
\setcounter{metacounter}{0}
\gdef\exb@emptyList{0}
\newif\ifexb@tagbuildmode
\global\exb@tagbuildmodefalse
\global\let\emptyList\exb@emptyList
\global\let\exb@tagList\exb@emptyList
%
\gdef\ifppMode#1{
  \def\mname{\exbank@macroname{#1}}
  \strif\mname\ppMode
}
\gdef\isppMode#1#2{
  \edef\mname{\exbank@macroname{#1}}
  \strif{\exbank@macroname{#1}}{\ppMode}\relax#2\fi
}
% \ppMode is what list mode we're in. Either it's \pm@Exclude or \pm@Select
\gdef\@ifppMode#1{%
\ifnum\pdfstrcmp{\exbank@macroname{#1}}{\ppMode}=\z@%
\expandafter\@firstofone%
\else%
\expandafter\@noneofone%
\fi%
}%
\def\pm@Exclude{exclude}
\def\pm@Select{select}
\def\pm@Normal{normal}

%% Deciding whether problem (and accompanied solution) should be shown
\let\ipm\@ifppMode
\let\T@\isTrue
\let\F@\isFalse

% Short hands for doing pp mode selection:
\def\exb@DPT{\global\exb@DisplayCurrentProblemtrue}
\def\exb@DPF{\global\exb@DisplayCurrentProblemfalse}
\At\DecideProblemDisplay{%
    \ifnum\pdfstrcmp{\ppList}{\emptyList}=\z@\relax%
      \ipm\exclude\exb@DPT\ipm\select\exb@DPF
    \fi
    \exb@int@isin{\themetacounter}{\ppList}{
      \ipm\exclude\exb@DPF\ipm\select\exb@DPT\ipm\normal\exb@DPT
    }{
      \ipm\exclude\exb@DPT\ipm\select\exb@DPF\ipm\normal\exb@DPT
    }
\ifexb@tagbuildmode
\ifexb@DisplayCurrentProblem\relax%
\exb@intersection@any{\exb@currentTags}{\exb@tagList}{\global\exb@opthidesfalse}{\global\exb@opthidestrue\exb@DisplayCurrentProblemfalse}%
\fi
\fi
}

\gdef\introarg{\@isFalse}
%% Makeset arg handler
\pgfkeys{
 /makeset/.is family, /makeset,
 default/.style = {noheadarg=\@isFalse},
 intro/.style = {introarg=\@isTrue},
 nointro/.style = {introarg=\@isFalse},
 nohead/.style = {noheadarg=\@isTrue},
 head/.style = {noheadarg=\@isFalse},
 introarg/.estore in = \introarg,
 noheadarg/.is if = {exb@noheadArg},
}
%:=\makesetdefaults{nohead|intro}
%: Sets the default arguments to all makeset-keys.
%:e.g.
%:!\makesetdefaults{intro}
%:Will effectively make all
%:!\makeset{...}{...}
%:into
%:!\makeset[intro]{...}{...}
%:However, you can override this;
%:!\makesetdefaults{intro}
%:!\makeset[nointro]{...}{...}
%:(the \oarg{nointro} overrides the default \meta{intro} setting)
%:-
\gdef\makesetdefaults#1{%
\edef\exb@setpgfkeys{{\unexpanded\expandafter{/makeset/default/.style = {#1}}}}%
\expandafter\pgfkeys\exb@setpgfkeys%
}
\pgfkeys{/makeset/override/.cd,
intro/.code = {\gdef\introarg{\@isTrue}\ea\gdef\csname setlist@\exb@currentSetID @intro\endcsname{\@isTrue}},
nointro/.code = {\gdef\introarg{\@isFalse}\ea\gdef\csname setlist@\exb@currentSetID @intro\endcsname{\@isFalse}},
nohead/.code = {\exb@noheadArgtrue\ea\gdef\csname setlist@\exb@currentSetID @nohead\endcsname{\@isTrue}}
}
% When a \makeset is made, it adds the set name to the
% \@listOfSets for use in \sprite
\gdef\@listOfSets{}
\gdef\@spriteMode{\isFalse}
\gdef\exbank@spriteSets{\emptyList}
%:=\spritesets{setlist}
%: This is a command that is used by sprite to determine what sets are shown in sprite. If this is not given, sprite uses all sets given in \makset
%:-
\long\gdef\spritesets#1{
  \gdef\exbank@spriteSets{#1}
}
%% Generates sets that can be iterated over
%% \makeset[intro|nohead]{\select{path/to/file}{1,2}, \exclude{file}{1,2}, file}
%% Where file is a file in the \setExercisesDir-folder (default it's the same folder you have the file). See  and 1,2 are the part problem numbers
%:=\makeset[intro,nohead]{filable}
%: This command is the one you use to make a set! Later you use \buildset to build the sets you make. The \meta{filable} argument is either the name of the file relative to the \setExercisesDir-path (default is nothing, so it's in the root path), or you could use the \select or \exclude to  respectively cherry pick or exclude exercises. (See their docs).\\
%: \oarg{intro} this counts the intro environment as a part problem, so that you can \select or \exclude the intro\\
%: \oarg{nohead} prevents the builder from adding a problem header. This is handy if you want to create an exercise that is composed of multiple parts. You can use the \phead to insert the problem header where you want it
%:! \makeset[nohead]\{\phead, \select{myexercise}{1,2,3}}
%:-
\global\let\exlist@protect\noexpand
\newcommand\makeset[3][]{
  \xdef\thissetid{#2}
  \ea\xdef\csname setkeys@\thissetid\endcsname{#1}
  \pgfkeys{/makeset, default, #1}%
  % Is intro sent?
  \if\introarg\isTrue%
    \gdef\introarg{\isTrue}
    \ea\gdef\csname setlist@#2@intro\endcsname{\isTrue}
  \fi
  \ifexb@noheadArg%
    \ea\gdef\csname setlist@#2@nohead\endcsname{\isTrue}
  \fi
  \ea\gdef\csname setlist@#2\endcsname{\exlist@protect{#3}}
  \def\setmacro{\unexpanded\expandafter{\csname setlist@#2\endcsname}}
  % \g@addto@macro\@listOfSets{\csname setlist@#2\endcsname,}
  \g@addto@macro\@listOfSets{,#2}
}
%:=\about{text}
%:This contains information about an exercise set. It is intended to be on the top of an
%:exercise, explaining short what the exercise is about. It's only visible when using \sprite
%:-
%% It gets defined in \sprite when use
\long\gdef\about#1{}

%:=\sprite[PiP]
%: This is a way to visualize all exercises. It takes one optional argument which is how many pages
%: inside one page. Defaults to 4
%: \!begin{marker}If \sprite is used, it should be the only command in \begin{document}\end{document}\!end{marker}
%:-
\newcommand\sprite[1][4]{
% \exercisebanksetup{part problem header={\Large\textbf{\thepartproblemcounter}}}
\squeeze
\gdef\@spriteMode{\isTrue}
\long\def\about##1{{\Large\textbf{About}:\\[1.1em]##1\\[1.5em]}}
\pgfpagesuselayout{#1 on 1}[a4paper,border shrink=5mm]
\ifnum\pdfstrcmp{\exbank@spriteSets}{\emptyList}=\z@\relax%
  \edef\sprite@setlist{\ea\@secondoftwo\@listOfSets}
  %readlist with star removes whitespaces in the items
  \readlist*\list@sprite@setlist\sprite@setlist
  \foreachitem\set\in\list@sprite@setlist{
    \edef\theset{{\set}}
    \if\theset\empty\relax\else%
    \buildset{\set}%
    \fi%
  }
\else
  \edef\sprite@setlist{\exbank@spriteSets}
  \readlist*\list@sprite@setlist\sprite@setlist
  \foreachitem\set\in\list@sprite@setlist{
    \edef\theset{{\set}}
    \if\theset\empty\relax\else%
    \buildex{\set}%
    \fi%
  }
\fi
}
%:§ref
%:=\exerciseFile
%:This is a `read-only' macro that contains the name of the current exerciseFile
%:-
%:§makingSets
%!TEX root = main.tex
%:§makingSets
%% Sets the variables \ppList and \exerciseFile based on the current set
\gdef\exbank@setEnv@normal#1{
\exbank@setEnv{#1}{normal}
}
\gdef\exerciseFile{}%
\newcommand{\exbank@setEnv}[3][{-1}]{
  \if\@spriteMode\isFalse%
    \gdef\ppList{#1}
    \gdef\exerciseFile{#2}
    \gdef\ppMode{#3}
  \else
    \gdef\ppList{\emptyList}
    \gdef\exerciseFile{#2}
    \gdef\ppMode{\pm@Normal}
  \fi

}
%:=\exec{macros}
%:You can use this in \makeset to execute commands between problems.
%:E.g. to insert a new page in between two problems when using nohead:
%:!\makeset[nohead]{2}{\phead,my/exercise,\exec{\clearpage},next/exercise}
%:-
\global\let\ex@protect\noexpand
\long\gdef\exec#1{:\ex@protect{#1}}
% Documented as \DeclareExerciseCommand
\long\gdef\exb@def@makeset@cmd#1{\@ifnextchar[{\@exb@def@makeset@cmd{#1}}{\@exb@def@makeset@cmd{#1}[0]}}
\long\gdef\@exb@def@makeset@cmd#1[#2]#3{
\def\ncArgs{#1[#2]}%
\bgroup\globaldefs=1\ea\newcommand\ncArgs{\exec{#3}}\egroup%
}
% For backwards compability
\def\exbank@def@makeset@command{\exb@warn{\string\exbank@def@makeset@command\space is deprecated and will be removed in versions >= 0.3.0. Use \string\exb@def@makeset@cmd}\exb@def@makeset@cmd}
%:=\DeclareExerciseCommand{command}[numargs]{actions}
%:Now, use similar to newcommand. Does not support default arguments yet, but plans to.
%:The old way of defining still works:
%:!\DelcareExerciseCommand{\pbreak}\brackets{\clearpage}
%:and
%:!\DelcareExerciseCommand{\ptitle}[1]\brackets{\Large\textbf{#1}}
%:This can be used later in makesets. E.g.
%:\DelcareExerciseCommand{\pbreak}\brackets{\clearpage} will make
%:\pbreak behave like \clearpage in the set:
%:!\makeset[nohead]{%
%:! \phead,
%:! myExercise,
%:! \pbreak,
%:! myExerciseOnNewPage%
%:!}
%:-
\let\DeclareExerciseCommand\exb@def@makeset@cmd


%% Converts {A}{1,2,3} into [{1,2,3}]{A}{exclude} such that it can be sent
%% as optional arguments to exbank@setEnv. The last is set as ppMode (exclude/select/mix)
%:=\exclude{exerciseFileName}{Comma separated numbers}
%:As you can see in the intro section of the documentation, this is for excluding partproblems
%:To be used in \refCom{makeset}
% :-
\newcommand{\exclude}[2]{[{#2}]{#1}{exclude}}

%:=\select{exerciseFileName}{Comma separated numbers}
%:As you can see in the intro section of the documentation, this is for cherry picking partproblems
%:To be used in \refCom{makeset}
% :-
\newcommand{\select}[2]{[{#2}]{#1}{select}}
%:=\orderedselect
%:-
\newcommand{\orderedselect}[2]{%
\readlist*\thelist\csv@list%
\foreachitem\items\in\thelist{[{\item}]{#1},}%
}

%% Commands that can be sent to makeset
%% Problem header
\let\exbank@isFirstProblem\isTrue
\exb@def@makeset@cmd{\phead}{%
  \if\exbank@isFirstProblem\isTrue%
    \let\exbank@isFirstProblem\isFalse%
  \else%
    \Trigger\EndProblem%
  \fi%
% \ifhmode\else\par\fi
\par
  \Trigger\BeginProblem%
  \Trigger\VeryBeginProblem%
}
% Problem header with custom problem "number"
\exb@def@makeset@cmd{\pheadarg}[1]{
  \if\exbank@isFirstProblem\isTrue%
    \let\exbank@isFirstProblem\isFalse%
  \else%
    \Trigger\EndProblem%
  \fi%
  \Trigger\BeginProblem%
  \bgroup\def\theproblemcounter{#1}\Trigger\VeryBeginProblem\egroup%
}
\exb@def@makeset@cmd{\pbreak}{\clearpage}
%:-

%---- Protectors: ----%
%% \ex@protect:
%% \def\hello{\exec{\textbf{world}}} ~> :\noexpand{\textbf{world}}
%% Then when it's tested
%% \let\ex@protect\unexpanded
%% \edef\hello{\hello} ~> :\textbf{world}

%% Build one exercise
\gdef\buildex#1{
  \makeset{#1}{#1}
  \buildset{#1}
}
%%Logic for building a set
\exb@CountIntrosfalse
\gdef\exb@setpgf#1{%
\edef\keystring{{/makeset,default,#1}}%
\ea\pgfkeys\keystring}
%:=\buildset[intro|nohead|nointro]{setname}
%:This command runs the set given. The set has do be defined by \makeset. E.eg
%:!\makeset{myExerciseSet}{exercisefile1, \select{exercisefile2}{1,2}}
%:!\!begin{document}
%:!  \buildset{myExerciseSet}
%:!\!end{document}
%:-
\newcommand\buildset[2][]{%
  \xdef\exb@currentSetID{#2}%
  \pgfkeys{/makeset/override/.cd, #1}
  \global\let\setName\exb@currentSetID%
  \gdef\exb@buildset@oarg{#1}
  \@ifundefined{setlist@#2@intro}{\exb@CountIntrosfalse}{\exb@CountIntrostrue}%
  \@ifundefined{setlist@#2@nohead}{\gdef\nohead{\isFalse}}{\gdef\nohead{\isTrue}}%
  %% Setting the global setName
%:=\setName
%:This variable prints out the name of your set that you sent to \buildset.
%:The following example prints "Exercise set number 1" and "Exercise set number 2" on the top of each set
%:!\At\StartBuildset{
%:! Exercise set number \setName
%:!}
%:! %... \makesets here ...%
%:!\begin{document}
%:! \buildset{1}{myexercise}
%:! \buildset{2}{myexercise}
%:!\end{document}
%:-
  % Set keys for this set in case user set some keys
  \ifcsname setkeys@\setName\endcsname%
  \xdef\skeys{\csname setkeys@#2\endcsname}%
  % \xdef\skeys{default, \skeys}
  \exb@setpgf\skeys%
  \fi%
  %% Warnins if the set doesn't exist
  \@ifundefined{setlist@#2}{%
    \exb@err{Couldn't find set #2. Did you remember to do \string\makeset{#2}{}? }
    \stop\bye
  }{}
  %% Sets the current setlist from the variable
  %% generated in makeset
  \let\exlist@protect\unexpanded%
  \edef\exbank@setlist{\csname setlist@#2\endcsname}%
  \let\exlist@protect\noexpand%
  \if\@spriteMode\isFalse%
    \if\exb@frontpage\@isFalse\else\input{\exb@frontpage}\fi
    \Trigger\StartBuildset%
  \fi%
  % ---- BEGIN setlist divider ----%
  % The following code distinguish between the different things that can be sent to \makeset.
  % At time of writing that is:
  % \select, \exclude, \exec, or ExerciseCommands: e.g. \pbreak and \phead (included customs)
  % \select and \exclude expands to [#2]{#1}{select|exclude} i.e. \exclude{file}{1,2,3} -> [1,2,3]{file}{exclude}
  % and are se to exbank@setEnv for parsing. The last expanded "argument" is the \@ppmode
  % or exclude / select eg [1,3]{some,ex}{select}
  % makesetcommands are expanded to ?{@somecommand}, where @somecommand is to be executed.
  % e.g. \phead -> ?{@phead} and the code runs \@phead at appropriate time.
  % For \exec, it expands to :{#1}, and the code runs #1 at appropriate time.
  % e.g. \exec{\textbf{HELLO}} -> :{\textbf{HELLO}}.
  % So if the first character of the current item in the list is a ? or :
  % then execute makeset@command or exec respectively.
  % However, if it's a [, then \select or \exclude is invoked and send this to exbank@setEnv with the args.
  % else it's a normal filename and should be sent as \exbank@setEnv{THEFILENAME}{normal}, where normal
  % tells the rest of the program that all exercises should be included.
  %
  % Is it a makeset@command?
  \gdef\@delegateFileInfo{\@ifnextchar?\@execute@makeset@command\@is@exec}%
  % Is it an exec command?%
  \def\@is@exec{\@ifnextchar:\@execute@exec\@is@file}%
  % Is it a \select/\exclude or is it just a filename?%
  \gdef\@is@file{\@ifnextchar[\exb@setEnv@withOptargs\X@exb@setEnv}%
  \gdef\X@exb@setEnv##1{%
    \def\continueLoop{\isTrue}%
    \@dinfo{Processing normal file "\exerciseFileInfo.tex"}%
    \exbank@setEnv@normal{\exerciseFileInfo}\bgroup\nullfont%
  }
  \gdef\exb@setEnv@withOptargs{%
    \def\continueLoop{\isTrue}\ea\exbank@setEnv\exerciseFileInfo\bgroup\nullfont%
  }

  \gdef\@execute@makeset@command ?##1{%
    \edef\inner{\@firstofone##1}%
    \@dinfo{Executing macro \@backslashchar\@gobble##1}%
    \csname \inner\endcsname%
    #1\def\continueLoop{\isFalse}\ea\bgroup%
  }
  \def\@execute@exec:{%
    \def\continueLoop{\isFalse}\@dinfo{Executing custom command}\bgroup%
  }
%% Loop through list
  \readlist*\setlist\exbank@setlist%
  \foreachitem\exerciseFileInfo\in\setlist{%
    \def\continueLoop{\isTrue}%
    %% ppList is \emptyList (0) if it's not a list
    \gdef\ppList{\emptyList}%
    %% Counter used for iterating over partproblems
    %% (Cant' use partproblemcounter since some of them might be excluded)
    \setcounter{metacounter}{0}%
    %% Checking whether we have optional args
    %% I.e. one of the arguments are sent via \exclude
    %% And then send it to \exbank@setEnv
    %If \exerciseFileInfo starts with a [, then optional arguments
    %that are to be sent to \exbank@setEnv. That is from \exclude or \select
    \let\ex@protect\unexpanded%
    \edef\exerciseFileInfo{\exerciseFileInfo}%
    \ea\@delegateFileInfo\exerciseFileInfo\egroup%
    \Trigger\InputExercise%
    \if\continueLoop\isTrue%
    \let\ex@protect\noexpand%
%% TRIGGER EVERTYHING
  \Trigger\PathControl%
%:$triggers.=\Trigger\InputExercise:\\ Triggers before a file is included\\\
%:$triggers.=\Trigger\BeginProblem:\\ Triggers before a file is included, but only if problem headers are to be written (no [nohead] given)\\
%:$triggers.=\Trigger\EndProblem:\\ Triggers right after problem is included if [nohead] \emph{not} given\\
%:$triggers.=\Trigger\BeginBuildset:\\ Triggers right before a set has begun building (not if \dac{sprite} is used). You might want to put your set-header here\\
%:$triggers.=\Trigger\EndBuildset:\\ Triggers when a set has stopped building (not if \dac{sprite} is used)\\
    \if\nohead\isFalse%
      \if\@spriteMode\isFalse%
        \Trigger\BeginProblem%
      \fi%
    \fi%
    %% \VeryBeginProblem is only for stuff that
    %% is intended to be a part of the problem
    %% e.g. problem header is using this.
    \if\nohead\isFalse\Trigger\VeryBeginProblem\fi%
    \if\@spriteMode\isTrue\textbf{\exerciseFile.tex\\}\fi%
    %% \incl = \ input because otherwise latexpanded would try.
    %% to input the file
    \IfFileExists{\@exercisesDir/\exerciseFile}{%
      \Trigger\InputExerciseFile
      \incl{\@exercisesDir/\exerciseFile}%
      \if\@spriteMode\isTrue%
        \setcounter{partproblemcounter}{0}%
      \fi%
    }{
      \@latex@error{Could not find \@exercisesDir/\exerciseFile. Maybe it is because the default exercise directory is now changed to the same directory that your main file is in. To set default exercise directory to exercises, do \string\exercisebanksetup{exercise directory=exercises}}{}
      \stop\bye
    }
    %% More triggers
    \if\nohead\isFalse\Trigger\EndProblem\fi%
    \fi%<- End \continueLoop
  }
  \if\@spriteMode\isFalse%
    \Trigger\EndBuildset%
  \fi%
  \setcounter{problemcounter}{0}%
  \setcounter{partproblemcounter}{0}%
  \clearpage%
}
%:=\buildsets[intro|nohead|nointro]{list}
%:This command will generate multiple sets:
%:!\buildsets{set,set2,set3}
%:-
\newcommand\buildsets[2][]{%
	\xdef\csvlist{#2}%
	\readlist*\items\csvlist%
	\foreachitem\set\in\items{%
		\buildset[#1]{\set}%
	}
}
%:=\buildtags{tagslist}{setslist}
%:This command will generate multiple sets:
%:!\buildsets{tag,tag2,tag3}{set,set2,set3}
%:-
\newcommand\buildtags[2]{%
  \global\exb@tagbuildmodetrue
  \xdef\csvtags{#1}%
  \xdef\csvlist{#2}
	\xdef\csvsets{#2}%
	\readlist*\tagsItems\csvtags%
  \readlist*\setsItems\csvsets%
	\foreachitem\set\in\setsItems{%
    \xdef\exb@tagList{\csvtags}
		\buildset{\set}%
	}
  \global\exb@tagbuildmodefalse
}

%!TEX root = ../main.tex
%% Reference problems
\newcommand\refcounter[1]{
\edef\@currentlabel{#1}%
}
\DeclareRobustCommand{\pplabel}[1]{
	\refcounter{\theproblemcounter}\label{pr:\exerciseFile:#1}
	\refcounter{\thepartproblemcounter}\label{pp:\exerciseFile:#1}
}
\let\pptag = \pplabel

\newcommand\pppref[1]{%
(\ref{pp:\exerciseFile:#1})\relax%
}
\newcommand\ppref[1]{%
\ref{pr:\exerciseFile:#1}\ref{pp:\exerciseFile:#1}\relax%
}
\newcommand\pref[1]{%
\ref{pr:\exerciseFile:#1}\relax%
}


\makeatother
\let\dac\docAuxCommand
\long\def\keyDef#1#2#3#4{\begin{docKey}{#1}{=\meta{#2}}{\meta{default}=#3}#4\end{docKey}}
\tcbset{documentation listing style=mydocumentation}
% Magenta HREF style
\let\oldhref\href
\gdef\href#1#2{{\color{magenta}\oldhref{#1}{#2}}}
% Give section some space
\let\oldsection\section
\gdef\section{\needspace{0.3\paperheight}\oldsection}
\let\oldsubsection\subsection
\gdef\subsection{\needspace{0.2\paperheight}\oldsubsection}


\setlength{\parindent}{0pt}
\title{{@@PACKAGE - manual\\ @@VERSION{\\[-0.5em]\footnotesize(build @@BUILD)}}}
\author{Andreas Strauman}
\begin{document}
\maketitle
\input{pretoc}

If you found any bugs or want new functionality, to contribute, view the commented source, get latest version of this package or get in touch with me, you can do all of that at \url{@@GITHUB}. If you have questions of functionality, kindly direct them to the community\\ \url{http://tex.stackexchange.com}. The author is active on this site regularly.

\tableofcontents
\clearpage
\section{Motivation}
Exercises are saved as separate files containing part problems. These files can be used to make sets, and you can cherry-pick or exclude certain part problems as you see fit. This makes it easier to maintain and keep your exercises flexible as the syllabus changes.

\section{Flow/Moderate start}
I suspect that working with this package will break you current flow. So let's go throught it.

This package assumes you put all of your exercises within the folder named \texttt{exercises} (you can change the default folder using \refCom{setExercisesDir})
\begin{dispListing*}{title=exercises/myexercise.tex}
\begin{intro}
  This introduces our problem
\end{intro}
\begin{problem}
  This is a partproblem 1,
  and will be hidden (just wait, you'll see)
\end{problem}
\begin{problem}
  This is a partproblem 2.
  This will not be hidden, but become part problem a!
\end{problem}
\end{dispListing*}
Let's build all of them first. In the main file, (the one where you include this package):
\begin{dispListing*}{title=main.tex}
  \documentclass{article}
  \usepackage{exbank}
  \makeset{myExerciseSet}{myexercise}
  \begin{document}
    \buildset{myExerciseSet}
  \end{document}
\end{dispListing*}
This builds the entire set, and adds Problem header and partproblem counters ( (1a) and (1b) ) by default.
\subsection{Select}
Now, let's build only the second problem.
\begin{dispListing*}{title=main.tex}
  \documentclass{article}
  \usepackage{exbank}
  \makeset{myExerciseSet}{\select{myexercise}{2}}
  \begin{document}
    \buildset{myExerciseSet}
  \end{document}
\end{dispListing*}
This should only build the intro and the one exercise you \refCom*{select}ed!

Now, say you want to hide the intro. Well all you have to do in this case is
make the package treat the intro as a problem in regards to what is \refCom*{select}ed.
Just add the optional argument \oarg{intro} to \dac{make}. That is switch
\begin{dispListing}
\makeset{myExerciseSet}{\select{myexercise}{2}}
\end{dispListing}
with
\begin{dispListing}
\makeset[intro]{myExerciseSet}{\select{myexercise}{3}}
\end{dispListing}
Notice that there are 3 `partproblems' now since we have to count the intro!

\subsection{Exclude}
But what if you have an exercise with 12 partproblems, and you only want to exclude the 7th partproblem? Well, then \dac{Exclude} is here to rescue the day for you.
\begin{dispListing}
  \makeset{myExerciseSet}{\exclude{soManyExercises}{7}}
\end{dispListing}
Here it's important to note that the [intro] argument would not make the intros disappear. If we wanted to only exclude the intro from our previous example file \texttt{exercises/myexercise.tex} we would do
\begin{dispListing}
\makeset[intro]{myExerciseSet}{\exclude{myexercise}{1}}
\end{dispListing}
So we're excluding the partproblem 1. But that's the intro when we send the [intro] optional argument
\subsection{Multiple}
In \dac{makeset} you can just separate exercises with commas! Here is an example:

Let's say you have two files with exercises. One located in \texttt{exercises/circuits/RLC.tex} and one in \texttt{exercises/ohm/ohmsGeneralLaw.tex}, and you want to include partproblem 1 through 5 from \texttt{RLC.tex} and all of the exercises from \texttt{ohmsGeneralLaw.tex}.

\begin{dispListing}
\makeset{\select{circuits/RLC}{1,...,5}, ohmsGeneralLaw}
\end{dispListing}
This will divide it up with problem headers. So that what is in the \texttt{RLC.tex}-file will be Problem 1, and \texttt{ohmsGeneralLaw.tex} Problem 2.
\subsection{Mixnmatch}
What if you want to make both of them the same exercise? Well, then you pass the [nohead] argument to \dac{makeset}:
\begin{dispListing}
\makeset[nohead]{\phead, \select{circuits/RLC}{1,...,5}, ohmsGeneralLaw}
\end{dispListing}
The \dac{phead} command makes a problem header. You can pass them as much as you want:

\begin{dispListing}
\makeset[nohead]{\phead, \select{circuits/RLC}{1,...,5},
              ohmsGeneralLaw, \phead, someOtherExercise, moreExercises}
\end{dispListing}

\subsection{Solutions}
The last thing to cover then is solutions. In your exercise files you just use the solution environment
\begin{dispListing}
\begin{solution}
Solution goes here
\end{solution}
\end{dispListing}
They are hidden by default, so you would have to use \dac{DisplaySolutions} in
your main file to display them.

That covers the basics. Enjoy
\begin{marker}
\dac{begin}\{problem\},\dac{end}\{problem\},\\
\dac{begin}\{solution\},\dac{end}\{solution\},\\
\dac{begin}\{intro\} and \dac{end}\{intro\} has to be on their own line without any spaces!
\end{marker}

\section{Reference}
\subsection{Environments}
\begin{docEnvironment}{problem}{}
Inside the \dac{keyRef}\{exercise directory\}, you keep your exercises. Inside the exercise file you'd use a problem environment to write your partproblems. It might be a little confusing that you're using \dac{begin}\{problem\} instead of \dac{begin}\{partproblem\} when you're writing a partproblem, but it's less typing.
\end{docEnvironment}
\begin{docEnvironment}{solution}{}
 Things inside here is only visible if \refCom{DisplaySolutions} are given before \dac{begin}\{document\}
 \begin{marker}\dac{end}\{solution\} has to be on it's own line without any leading spaces!\end{marker}
\end{docEnvironment}
\begin{docCommand}{DisplaySolutions}{}
Turns on the solutions, so they are shown.
\end{docCommand}
\begin{docEnvironment}{intro}{}
Sometimes you'd want to introcude your exercises and tell a little bit about it. Maybe have a figure there also. Those things should go inside this environment. This can be treated as a problem in terms of counting. See \refCom{makeset} for more info.
\end{docEnvironment}
\subsection{Configuration and options}
You can do a lot of configurations on this package, and probably
 even more to come in later versions!
\begin{docCommand}{exercisebanksetup}{\marg{[key/values]}}
 Here is a list of the different keys and their meaning
\keyDef{part problems}{On/Off}{On}{
 This is whether or not to do part problems. E.g. 1a), 1b) etc.
 If this is turned Off, then the part problems will be treated as problems
}
\keyDef{tighten paragraphs}{True/False}{True}{
 Disabling this will prevent
 the package from attempting to prevent part problems to scatter across pages
}
\keyDef{problem header}{macro}{see below}{
 This sets the problem header. To access the translation of the problem text, use \dac{@tr}\{Problem\}, and
 the problem counter is accessed with \dac{theproblemcounter}.\\
 Defaults to\\
 \brackets{\dac{normalfont}\dac{Large}\dac{bfseries}\dac{@tr}\{Problem\}~\dac{theproblemcounter}}.
}
\keyDef{part problem header}{macro}{see below}{
 This sets the problem header. To access the current problem, use \dac{theproblemcounter}, and then the current
 part problem \dac{thepartproblemcounter}. To make it a letter, as per default use \dac{alph}\{partproblemcounter\}\\
 Default is:\\
 \dac{large}\dac{textbf}\{(\dac{theproblemcounter}\dac{alph}{partproblemcounter\})}
}
\keyDef{solution header}{string}{see below}{
\dac{large}\{\dac{textbf}{\dac{@tr}{Solution\}:}}
}
\keyDef{exercise directory}{string}{./}{
This key is used for setting the default exercise directory.
}
\begin{dispListing}
 \exercisebanksetup{exercise directory=exercises,part problems=Off,solution header={\textbf{SOL:}}}
\end{dispListing}
\end{docCommand}
\subsection{Internationalization}
\begin{docCommand}{translateExBank}{\marg{Translation key/vals}}
This is to translate the text inside the package. As of now the available key/values are
\begin{itemize}
\item Problem
\item Solution
\end{itemize}
The Norwegian translation would then be done with
\begin{dispListing}
\translateExBank{Problem=Oppgave, Solution=Løsning}
\end{dispListing}
\end{docCommand}
\subsection{Triggers}
\begin{docCommand}{Trigger}{\marg{Any Macro}}
 See \refCom{At}\\
 Available triggers:\\
\dac{Trigger}\dac{BeginPartproblem}:\\ Triggers before a partproblem is inserted\\
\dac{Trigger}\dac{VeryBeginPartproblem}:\\ Triggers right after \dac{BeginPartproblem}. This is so that the user can do stuff before the actual headers start. The partproblem headers are invoked by \dac{At}\dac{VeryBeginPartproblem}
\dac{Trigger}\dac{InputExercise}:\\ Triggers before a file is included\\\
\dac{Trigger}\dac{BeginProblem}:\\ Triggers before a file is included, but only if problem headers are to be written (no [nohead] given)\\
\dac{Trigger}\dac{EndProblem}:\\ Triggers right after problem is included if [nohead] \dac{emph}\{not\} given\\
\dac{Trigger}\dac{BeginBuildset}:\\ Triggers right before a set has begun building (not if \dac{sprite} is used). You might want to put your set-header here\\
\dac{Trigger}\dac{EndBuildset}:\\ Triggers when a set has stopped building (not if \dac{sprite} is used)\\


\end{docCommand}
\subsection{General reference}
\begin{docCommand}{ownLineNoSpacesGotIt}{}
This is to annoy the user enough to get his attention about the requirements of the \refEnv{problem}, \refEnv{solution} and \refEnv{intro} environments.\\
\end{docCommand}
\begin{marker}DEPRECATED! use \refCom{exercisebanksetup} with \refKey{exercise directory} instead!\end{marker}
\begin{docCommand}{setExercisesDir}{\marg{directory}}
\begin{marker}\dac{setExercisesDir} is deprecated! use \refCom{exercisebanksetup} with \refKey{exercise directory} instead!\end{marker}
This is the directory, relative to the file you included the package,
where the package should be looking for exercises. Default is the same directory as your main file (the one you build).
\end{docCommand}
This package also includes some extra stuff. For example the \dac{At} and \dac{Trigger}
\begin{docCommand}{At}{\marg{AnyMacro}}
Here you can send any macro because it isn't evaluated! For example \dac{At}\dac{BeginSomething} is fine and even if \dac{BeginSomething} is not defined. Also and when using \dac{Trigger} it just ignores it if it didn't exist. It's pretty similar in function as to \dac{AtBeginDocument}.
\begin{dispListing}
 \At\BeginSomething{DoSomething}
 Which is triggered with
 \Trigger\BeginSomething, this evaluates to DoSomething
\end{dispListing}
\end{docCommand}
\begin{docCommand}{isin}{\marg{haystack}\marg{needle}\marg{True}\marg{False}}
 \meta{haystack} is a comma separated list of integers\\
 \meta{needle} is an integer\\
 Executes \meta{True} if \meta{needle} is found in \meta{haystack}
 else executes \meta{False}
\end{docCommand}
\begin{docCommand}{exerciseFile}{}
This is a `read-only' macro that contains the name of the current exerciseFile
\end{docCommand}
\subsection{Making sets}
\begin{docCommand}{makeset}{\oarg{intro,nohead}\marg{filable}}
 This command is the one you use to make a set! Later you use \dac{buildset} to build the sets you make. The \meta{filable} argument is either the name of the file relative to the \dac{setExercisesDir}-path (default is nothing, so it's in the root path), or you could use the \dac{select} or \dac{exclude} to  respectively cherry pick or exclude exercises. (See their docs).\\
 \oarg{intro} this counts the intro environment as a part problem, so that you can \dac{select} or \dac{exclude} the intro\\
 \oarg{nohead} prevents the builder from adding a problem header. This is handy if you want to create an exercise that is composed of multiple parts. You can use the \dac{phead} to insert the problem header where you want it
\begin{dispListing}
 \makeset[nohead]\{\phead, \select{myexercise}{1,2,3}}
\end{dispListing}
\end{docCommand}
\begin{docCommand}{about}{\marg{text}}
This contains information about an exercise set. It is intended to be on the top of an
exercise, explaining short what the exercise is about. It's only visible when using \dac{sprite}
\end{docCommand}
\begin{docCommand}{sprite}{\oarg{PiP}}
 This is a way to visualize all exercises. It takes one optional argument which is how many pages
 inside one page. Defaults to 4
 \begin{marker}If \dac{sprite} is used, it should be the only command in \dac{begin}\{document\}\dac{end}\{document\}\end{marker}
\end{docCommand}
\begin{docCommand}{exclude}{\marg{exerciseFileName}\marg{Comma separated numbers}}
As you can see in the intro section of the documentation, this is for excluding partproblems
To be used in \refCom{makeset}
\end{docCommand}
\begin{docCommand}{select}{\marg{exerciseFileName}\marg{Comma separated numbers}}
As you can see in the intro section of the documentation, this is for cherry picking partproblems
To be used in \refCom{makeset}
\end{docCommand}
\begin{docCommand}{exec}{\marg{macros}}
You can use this in \dac{makeset} to execute commands between problems.
E.g. to insert a new page in between two problems when using nohead:
\begin{dispListing}
\makeset[nohead]{2}{\phead,my/exercise,\exec{\clearpage},next/exercise}
\end{dispListing}
\end{docCommand}
\begin{docCommand}{setName}{}
This variable prints out the name of your set that you sent to \dac{buildset}.
The following example prints "Exercise set number 1" and "Exercise set number 2" on the top of each set
\begin{dispListing}
\At\StartBuildset{
 Exercise set number \setName
}
 %... \makesets here ...%
\begin{document}
 \buildset{1}{myexercise}
 \buildset{2}{myexercise}
\end{document}
\end{dispListing}
\end{docCommand}
\begin{docCommand}{pplabel}{\marg{label}}
Labels a partproblem. You can reference to it later using \dac{ppref}\{\meta{label\}}
\end{docCommand}
\begin{docCommand}{ppref}{\marg{label}}
Reference a partproblem created by \dac{pplabel}\{\meta{label\}}. This prints e.g. 1c)
\end{docCommand}
\begin{docCommand}{ppref}{\marg{label}}
 Reference a partproblem created by \dac{pplabel}\{\meta{label\}}. This prints e.g. 1
\end{docCommand}
\subsection{Counters}
\docCounter{problemcounter}-counter holds the current problem number and
\docCounter{partproblemcounter}-counter holds the current partproblem \emph{number}.

\section{Changelog}
%!TEX root = exercisebank-doc.tex
\newversion{v0.0.2b11 2018/04/02}
  \change{Updated documentation syntax.}
\newversion{v0.0.3b38 2018/04/03}\nobreak
    \change{Updated triggers doc}
    \change{added trigger \dac{VeryBeginPartproblem}}

\newversion{v0.0.3b40 2018/04/03}
    \change{Added examples that uses the \refCom{At} command.}
    \change{Making front page and other snacks}

\newversion{v0.0.4b44 2018/04/03}
    \change{Changed design of part problems. (Looks much better now!)}

\newversion{v0.0.5b46 2018/04/03}
    \change{Fixed partproblems and solutions to fit on pages using \dac{filbreak}.}

\newversion{v0.1.0 2018/04/08}
    \change{Fixed title of documentation to match actual package.}
    \change{Fixed weird paragraph styling when displaying solutions}
    \change{Added a few package options. More to come!}

\newversion{v0.1.1 2018/04/13}
  \change{Added \refCom{exec}, which allows the user to execute macros between problems}
  \change{Bug fix: \refCom{At} would cause crash due to latexmk multiple builds}
  \change{Bug fix: \refCom{sprite} wouldn't build correctly}
  \change{Bug fix: misc bugs involving \refCom{makeset}, \refCom{buildset}, nohead and \refCom{phead}}

\newversion{v0.1.2 2018/04/17}
    \change{Fixed bug that \refCom{select} and \refCom{exclude} not working as expected.}
    \change{Added \refCom{ShowNumbers} for displaying numbers related to use in \refCom{select} and \refCom{exclude}}

\newversion{v0.1.3 2018/04/20}
    \change{Added custom dynamic figure path \refKey{figure root directory}.}
    \change{Fixed bugs related to \refCom*{phead} and the commands used in \refCom{makeset}-lists.}
\newversion{V0.1.4 2018/04/28}
    % \change{Introduced solutions only: \refCom{SolutionsOnly}}
    \change{Updated margins a lot!}
    \change{Fixed sneaky space in translation}
    \change{Fixed paragraph tightening when displaying solutions}
    \change{Introduced \refCom{DeclareExerciseCommand}}
    \change{Fixed bug with \refCom{exec}}
    \change{Iteration over items now uses a more lighweight approach.}
    \change{For commands that takes `lists' as arguments (\refCom{makeset}, \refCom{select} and \refCom{exclude}), the last item of the list can be terminated with new line or spaces without problems.}
    \change{For commands that takes `lists' as arguments, the entries in the list are now trimmed whitespaces on both sides.}
\newversion{v0.2.0 2018/07/21}
    \change{Fixed bug where \refCom{ppref} did not reference letter in part problem.}
    \change{Introducing \refCom{buildsets}-command for building more than one set at a time.}
    \change{Ability to add front page with the \refKey{front page} setup-key.}
    \change{You can now remove the forced margin setup with \refKey{style margins}.}
\newversion{v0.2.1 2018/09/24}
    \change{Added options for individual part problems using the \dac{nextproblem}-command}
    \change{Added tagging for part problems using using the \dac{nextproblem}-command and \dac{buildtags}-commands.}
    \change{Fixed problems related to \dac{ShowNumbers} not showing correctly.}
    \change{Fixed \dac{exec} so that it now can take paragraphs (made it \dac{long})}
    \change{Added point system}
    \change{Fixed bug where vertical space would remain when intro environment hidden}
    \change{\dac\DeclareExerciseCommand~now takes args}
    \change{Keys sent to \dac\makeset can now be sent to \dac\buildset~and~\dac\buildsets}
    \change{Introducing \dac\makesetdefaults}
\newversion{v0.2.2 2018/10/04}
    \change{Made trailing commas ignored in all lists (\dac\makeset,\dac\buildset,\dac\select,\dac\exclude, etc.)}
    \change{Fixed bug where tags wouldn't hide.}
\newversion{@@VERSION @@DATE}
    \change{Added possibility of custom part problem header from \dac\nextproblem}
    \change{Added possibility of showing problem only when \dac\DisplaySolutions active}
    \change{Corrected use of the length \dac\ppMarginBelow}
    \change{Fixed bug where \dac\phead and \dac\ShowFilenames crash}
    \change{Introduced \dac\SolutionsOnly}
    \change{Fixed bug where part problem header repeats if solution is itemize.}
    \change{Introduced the \dac\@rigid command for the \dac\At functionality. Also the \dac\ClearHook for deleting a hook}
    \change{\dac\buildset now gives error if set does not exist.}
    \change{Made \dac\HideTags cummulative.}
    \change{Created \dac\ShowAllTags for clearing tags list}
    \change{Raises error when pdfTeX(or pdfLaTeX) is not used}
    \change{Allow underscore for files in \dac\makeset}
    \change{Added \dac\exercisepoints for getting total number of points in an exercise/problem.}
    \change{added \dac\pgref and \dac\ppgref}
    \change{Margins now more manageable}
    \change{Prevent overwriting of \dac\graphicspath}
    \change{Set (append) to \dac\input@path no matter what.}
\chlogtable


% Testers and thanks
\thankyous{\gh{thorstengrothe}}
\tester{\gh{tristelune1}}

\end{document}
