%!TEX root = main.tex
\makeatletter
\edef\@isTrue{1}
\edef\@isFalse{0}
% This file contains the definition of the \isin macro
%:§ref
%:=\isin{haystack}{needle}{True}{False}
%: \meta{haystack} is a comma separated list of integers\\
%: \meta{needle} is an integer\\
%: Executes \meta{True} if \meta{needle} is found in \meta{haystack}
%: else executes \meta{False}
%:-
% #1: haystack
% #2: needle
% #3: action if found in list
% #4: action if not found in list
\gdef\isin#1#2#3#4{
 \def\needle{#2}
 \def\haystack{#1}
 \def\isFalse{0}
 \let\isInList = \isFalse
 % \IfInteger{\needle}{}{\@latex@error{ISIN: Can't look for part problem that is not numeric!}{}\bye\stop}
 \IfInteger{\haystack}{
  \ifnum\pdfstrcmp{\haystack}{\needle}=\z@\relax
   #3
  \else
   #4
  \fi
 }{
  \foreach \pp in #1{
    \ifnum\pdfstrcmp{\pp}{\needle}=\z@\relax%
     \gdef\isInList{1}
     #3
    \fi
   }
  \if\isInList\isFalse
   #4
  \fi
 }
}
\newif\ifexb@isInList
\gdef\exb@isin#1#2#3#4{
  \global\exb@isInListfalse
  \xdef\needle{#1}
  \xdef\haystack{#2}
  \def\whenfalse{#4}
  \def\whentrue{#3}
  \readlist*\items\haystack%
  \foreachitem\itm\in\items{
    \ifnum\pdfstrcmp{\needle}{\itm}=\z@\relax%
  \global\exb@isInListtrue%
  #3\fi
  }
  \ifexb@isInList\else%
  #4
  \fi%
}

%\strif: Compare two strings
%\strif\stringA\stringB%
%TRUE
%\else
%FALSE
%\fi
\gdef\strif#1#2{%
 \ifnum\pdfstrcmp{#1}{#2}=\z@\relax%
}
% Get macroname of macro
\let\ea = \expandafter
\begingroup\lccode`\|=`\\
\lowercase{\endgroup\def\removebs#1{\if#1|\else#1\fi}}
\gdef\exbank@macroname#1{\expandafter\removebs\string#1}
\gdef\@noneofone#1{}
\gdef\@noneoftwo#1#2{}

\@ifundefined{exbank@quiet}{
 \global\let\@dlog\@latex@warning
}{
 \global\let\@dlog\@noneofone
}

\newcommand\ex@before[2]{
\edef #1{\unexpanded{#2}\unexpanded\expandafter{#1}}
}
% Put stuff at left margin
% \exb@lmargin{content}{marginSize}
\newcommand{\exb@lmargin}[2]{\strut\vadjust{\exb@domark{#1}{#2}}}

\newcommand{\exb@domark}[2]{%
  \hbox to #2{
    \vbox to 0pt{
      \kern-\dp\strutbox
      \smash{\llap{#1}}
      \vss
    }%
  }
}
