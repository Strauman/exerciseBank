%!TEX root = main.tex
\makeatletter
% This file contains definition of custom labeling and referring.
%% Reference problems
%:§makingSets
%:-
\providecommand\refcounter[1]{
\edef\@currentlabel{#1}%
}
%:=\pplabel{label}
%:Labels a partproblem. You can reference to it later using \ppref{\meta{label}}
%:-
\DeclareRobustCommand{\pplabel}[1]{
	\refcounter{\theproblemcounter}\label{pr:\exerciseFile:#1}\label{pr:global:#1}
	\refcounter{\alph{partproblemcounter}}\label{pp:\exerciseFile:#1}\label{pp:global:#1}
}

\let\pptag = \pplabel
\newcommand\pppref[1]{%
(\ea\@alph\ref{pp:\exerciseFile:#1})\relax%
}
%:=\ppref{label}
%:Reference a partproblem created by \pplabel{\meta{label}}. This prints e.g. 1c)
%:-
\newcommand\ppref[1]{%
(\ref{pr:\exerciseFile:#1}\ref{pp:\exerciseFile:#1})\relax%
}
%:=\ppgref{label}
%:Reference a partproblem created by \pplabel{\meta{label}}. This can be used for \emph{global} exercises.
%:-
\newcommand\ppgref[1]{%
(\ref{pr:global:#1}\ref{pp:global:#1})\relax%
}
%:=\pref{label}
%: Reference a partproblem created by \pplabel{\meta{label}}. This prints e.g. 1
%:-
\newcommand\pref[1]{%
\ref{pr:\exerciseFile:#1}\relax%
}
%:=\pgref{label}
%: Reference a partproblem created by \pplabel{\meta{label}}. This prints e.g. 1, but have global search.
%:-
\newcommand\pgref[1]{%
\ref{pr:global:#1}\relax%
}
