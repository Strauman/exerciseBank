%!TEX root = main.tex
%:§problemopts
%:=\nextproblem
%:Prior to a problem environment in an exercise file you can pass options
%:using the \nextproblem command. If you e.g. wanted to hide one regardless
%:of what set it is you'd do
%:!\nextproblem{hide}
%:!\begin{problem}
%:!  This problem will be hidden because of the \nextproblem command
%:!\end{problem}
%:you can also tag problems using this:
%:!\nextproblem{tag=hard}
%:!\begin{problem}
%:!  This problem is hard
%:!\end{problem}
%:!\nextproblem{tag=hard}
%:!\begin{problem}
%:!  This problem is also hard
%:!\end{problem}
%:!\nextproblem{tag=easy}
%:!\begin{problem}
%:!  This problem is easy
%:!\end{problem}
%: And you could now build, say, only easy problems using
%:\buildtags{hard}\{SETNAME\}, where SETNAME is chosen by a
%:\makeset command.
%:-
%: You can also have multiple tags per exercise
%:!\nextproblem{tag={tag1,tag2}}
%:!\begin{problem}
%:!  This problem is easy
%:!\end{problem}
%:-
%:§problemopts
%:$problemKeys=Here are the \nextproblem keys:
%:$$problemKeys
%:-
%%% points are defined in poinstystem.tex
\newif\ifexb@opthides
\providecommand\exb@currentTags{}
\pgfkeys{
/exbank/problems/.cd,%
default/.style={hide=false,tag=\relax},%
% print/.code={Hello there! #1},
hide/.is if={exb@opthides},%
tags/.estore in=\exb@currentTags,%
tag/.code={\pgfkeysalso{tags={#1}}}%
% print/.code={\@dlog{#1}}
}
\def\nextproblem{\pgfqkeys{/exbank/problems}}
% \global\let\nextproblem\@gobble
