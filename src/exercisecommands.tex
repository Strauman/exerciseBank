%!TEX root = main.tex
%:§makingSets
%% Sets the variables \ppList and \exerciseFile based on the current set
\gdef\exbank@setEnv@normal#1{
\exbank@setEnv{#1}{normal}
}
\gdef\exerciseFile{}%
\newcommand{\exbank@setEnv}[3][{-1}]{
  \if\@spriteMode\isFalse%
    \gdef\ppList{#1}
    \gdef\exerciseFile{#2}
    \gdef\ppMode{#3}
  \else
    \gdef\ppList{\emptyList}
    \gdef\exerciseFile{#2}
    \gdef\ppMode{\pm@Normal}
  \fi

}
%:=\exec{macros}
%:You can use this in \makeset to execute commands between problems.
%:E.g. to insert a new page in between two problems when using nohead:
%:!\makeset[nohead]{2}{\phead,my/exercise,\exec{\clearpage},next/exercise}
%:-
\global\let\exb@execChar\relax
\global\let\ex@protect\noexpand
\long\gdef\exec#1{\exb@execChar\ex@protect{#1}}
% Documented as \DeclareExerciseCommand
\long\gdef\exb@def@makeset@cmd#1{\@ifnextchar[{\@exb@def@makeset@cmd{#1}}{\@exb@def@makeset@cmd{#1}[0]}}
\long\gdef\@exb@def@makeset@cmd#1[#2]#3{
\def\ncArgs{#1[#2]}%
\bgroup\globaldefs=1\ea\newcommand\ncArgs{\exec{#3}}\egroup%
}
% For backwards compability
\def\exbank@def@makeset@command{\exb@warn{\string\exbank@def@makeset@command\space is deprecated and will be removed in versions >= 0.3.0. Use \string\exb@def@makeset@cmd\space(or \string\DeclareExerciseCommand where applicable)}\exb@def@makeset@cmd}
%:=\DeclareExerciseCommand{command}[numargs]{actions}
%:Now, use similar to newcommand. Does not support default arguments yet, but plans to.
%:The old way of defining still works:
%:!\DelcareExerciseCommand{\pbreak}\brackets{\clearpage}
%:and
%:!\DelcareExerciseCommand{\ptitle}[1]\brackets{\Large\textbf{#1}}
%:This can be used later in makesets. E.g.
%:\DelcareExerciseCommand{\pbreak}\brackets{\clearpage} will make
%:\pbreak behave like \clearpage in the set:
%:!\makeset[nohead]{%
%:! \phead,
%:! myExercise,
%:! \pbreak,
%:! myExerciseOnNewPage%
%:!}
%:-
\let\DeclareExerciseCommand\exb@def@makeset@cmd


%% Converts {A}{1,2,3} into [{1,2,3}]{A}{exclude} such that it can be sent
%% as optional arguments to exbank@setEnv. The last is set as ppMode (exclude/select/mix)
%:=\exclude{exerciseFileName}{Comma separated numbers}
%:As you can see in the intro section of the documentation, this is for excluding partproblems
%:To be used in \refCom{makeset}
% :-
\newcommand{\exclude}[2]{[{#2}]{#1}{exclude}}

%:=\select{exerciseFileName}{Comma separated numbers}
%:As you can see in the intro section of the documentation, this is for cherry picking partproblems
%:To be used in \refCom{makeset}
% :-
\newcommand{\select}[2]{[{#2}]{#1}{select}}
%TODO: implement \orderedselect
%Like select, but cares about the order.
%
\newcommand{\orderedselect}[2]{%
% \readlist*\thelist\csv@list%
% \foreachitem\items\in\thelist{[{\item}]{#1},}%
}

%% Commands that can be sent to makeset
%% Problem header
\exb@isFirstProblemtrue
\exb@def@makeset@cmd{\phead}{%
  \ifexb@isFirstProblem%
    \global\exb@isFirstProblemfalse%
  \else%
    \Trigger\EndProblem%
  \fi%
% \ifhmode\else\par\fi
\par%
  \Trigger\BeginProblem%
  \Trigger\VeryBeginProblem%
}
% Problem header with custom problem "number"
\exb@def@makeset@cmd{\pheadarg}[1]{
  \ifexb@isFirstProblem%
    \global\exb@isFirstProblemfalse%
  \else%
    \Trigger\EndProblem%
  \fi%
  \Trigger\BeginProblem%
  \bgroup\def\theproblemcounter{#1}\Trigger\VeryBeginProblem\egroup%
}
\exb@def@makeset@cmd{\pbreak}{\clearpage}
%:-

%---- Protectors: ----%
%% \ex@protect:
%% \def\hello{\exec{\textbf{world}}} ~> :\noexpand{\textbf{world}}
%% Then when it's tested
%% \let\ex@protect\unexpanded
%% \edef\hello{\hello} ~> :\textbf{world}
