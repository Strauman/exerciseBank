%!TEX root = main.tex
\makeatletter
\edef\@isTrue{1}
\edef\@isFalse{0}
%:§config
%:You can do a lot of configurations on this package, and probably
%: even more to come in later versions!
%:=\ShowNumbers
%: Shows the numbers in fron of the part problems and intros that should be used with
%: \exclude and \select
%:-
\gdef\ShowNumbers{\gdef\@displayMetaCounter{\isTrue}}
%:=\ShowTags
%: Shows the tags in front of the part problems much like \ShowNumbers
%:-
\gdef\exb@showtags{\isFalse}
\gdef\ShowTags{\gdef\exb@showtags{\isTrue}}
\gdef\exb@printCurrentTags{Tags: \exb@currentTags~}
%:=\ShowFilenames
%: Shows the path+filename to the current exercise file
%:-
% \providecommand\exerciseFile{}
\gdef\ShowFilenames{\At\InputExerciseFile{\exerciseFile\par}}
%:§env
%:=\DisplaySolutions
%:Turns on the solutions, so they are shown.
%:-
\gdef\@DisplaySolutions{\@isFalse}
\gdef\DisplaySolutions{\global\exb@b@DisplaySolutionstrue\xdef\@DisplaySolutions{\isTrue}\@latex@warning{Showing solutions}}
\gdef\ifDisplaySolutions{\if\@DisplaySolutions\@isTrue}
%:=\SolutionsOnly
%:Displays only solutions (if there are any) to the part problem
%:-
\gdef\SolutionsOnly{\DisplaySolutions\global\exb@solutionsonlytrue}
\gdef\exb@currentPoints{0}
%:§config
%:-
% ---- \exercisebanksetup ----- %
\newif\ifexbank@if@needspace
\global\exbank@if@needspacetrue
\pgfkeys{
/exbanksetup/.is family, /exbanksetup,
default/.style = {
  part problems = On,
  tighten paragraphs = True,
  problem header={\normalsize\Large\bfseries\@tr{Problem}~\theproblemcounter},
  part problem header={\large\textbf{(\theproblemcounter\alph{partproblemcounter})}},
  part problem header suffix={},
  solution header={\large{\textbf{\@tr{Solution}:}}},
  exercise directory=.,
  figure root directory=\@exercisesDir,
  style margins=On,
  problem needs space=0.2\paperheight,
  no needspace/.is if={exbank@if@needspace},
  no needspace/.default = false
},
% Expanded macros
exercise directory/.estore in = \@exercisesDir, %<- excepted from opt-syntax
figure root directory/.estore in = \@figrootDir, %<- excepted from opt-syntax
front page/.estore in = \exb@frontpage,
front page=\@isFalse,
% Unexpanded macros
problem header/.store in = \exbank@opt@problemHeader,
part problem header/.store in = \exbank@opt@partProblemHeader,
solution header/.store in = \exbank@opt@solutionHeader,
problem needs space/.store in = \exbank@opt@problemneedspace,
part problem header suffix/.store in=\exb@partproblem@header@suffix,
% Bools
part problems/.style = {switches/#1/.get = \exbank@opt@partProblems},
tighten paragraphs/.style = {switches/#1/.get = \exbank@opt@tightenparagraphs},
style margins/.style={switches/#1/.get = \exbank@opt@doMargins},
% True/False
switches/.cd,
% True
  On/.initial = \@isTrue,
  on/.initial = \@isTrue,
  T/.initial =  \@isTrue,
  True/.initial =  \@isTrue,
% False
  Off/.initial = \@isFalse,
  off/.initial = \@isFalse,
  F/.initial = \@isFalse,
  False/.initial = \@isFalse,
}
%:=\exercisebanksetup{[key/values]}
\newcommand\exercisebanksetup[1]{
  \pgfkeys{/exbanksetup, #1}%
}
\exercisebanksetup{default}
\AtBeginDocument{\xdef\tpd{\the\prevdepth}}
\gdef\tighten@paragraph{%
\if\exbank@opt@tightenparagraphs\@isTrue\relax%
  % \if\@SolutionsOnly\@isFalse
    % \if\@DisplaySolutions\@isFalse
      \filbreak%
    % \fi
  % \fi
\fi%
}
\gdef\tighten@paragraph@solutions{%
\if\exbank@opt@tightenparagraphs\@isTrue\relax%
  % \if\@SolutionsOnly\@isFalse
    \if\@DisplaySolutions\@isTrue
      \filbreak%
    \fi
  % \fi
\fi%
}
\gdef\tighten@paragraph@always{%
\if\exbank@opt@tightenparagraphs\@isTrue\relax%
\filbreak%
\fi%
}


%: Here is a list of the different keys and their meaning
%:\keyDef{part problems}{On/Off}{On}{
%: This is whether or not to do part problems. E.g. 1a), 1b) etc.
%: If this is turned Off, then the part problems will be treated as problems
%:}
%:\keyDef{tighten paragraphs}{True/False}{True}{
%: Disabling this will prevent
%: the package from attempting to prevent part problems to scatter across pages
%:}
%:\keyDef{problem header}{macro}{see below}{
%: This sets the problem header. To access the translation of the problem text, use \@tr{Problem}, and
%: the problem counter is accessed with \theproblemcounter.\\
%: Defaults to\\
%: \brackets{\normalfont\Large\bfseries\@tr{Problem}~\theproblemcounter}.
%:}
%:\keyDef{problem needs space}{dimension}{0.2\pageheight}{
%:How much space has to be left on the page for a problem to start
%:}
%:\keyDef{no needspace}{bool}{false}{
%:Don't use \needspace command (only used in \refKey{problem needs space})
%:}
%:\keyDef{part problem header}{macro}{see below}{
%: This sets the problem header. To access the current problem, use \theproblemcounter, and then the current
%: part problem \thepartproblemcounter. To make it a letter, as per default use \alph{partproblemcounter}\\
%: Default is:\\
%: \large\textbf{(\theproblemcounter\alph{partproblemcounter})}
%:}
%:\keyDef{part problem header suffix}{macro}{\meta{empty}}{
%: This sets the suffix problem header, and defaults to be the number of points
%: the current exercise is worth.
%: Default is \meta{empty}
%:}
%:\keyDef{solution header}{string}{see below}{
%:\large{\textbf{\@tr\brackets{Solution:}}}
%:\@tr is the translation macro
%:}
%:\keyDef{exercise directory}{dir}{./}{
%:This key is used for setting the default exercise directory.
%:}
%:\keyDef{figure root directory}{dir}{\meta{exercise directory}}{
%:Exercisebank automatically allows you to \input{} and \includegraphics{} from
%:the same folder folder that your exercise is in, as well as a folder with the
%:same name as the exercise file. This is elaborated in the intro section.
%:However, you might want to put the figures inside a different directory.
%:For example if you have one directory containing your problems called \!texttt{exercises/},
%:set with \refKey{exercise directory}, and a file containing the exercise:
%: \!texttt{exercises/faradaysLaw/ACGenerate.tex}.
%:Then by default figures in the directories \!texttt{exercises/faradaysLaw/} and
%:\!texttt{exercises/faradaysLaw/ACGenerate/} can be included by just doing \includegraphics.
%:And if you want to change the root of this directory to be \!texttt{figures}, such that
%:exercisebank looks for figures in \!texttt{figures/faradaysLaw/} and \!texttt{figures/faradaysLaw/ACGenerate/}
%:You would use this option in the setup:\\
%:\exercisebanksetup{figure root directory=figures}
%:}
%:!\exercisebanksetup{
%:! exercise directory=exercises,
%:! part problems=Off,
%:! solution header={\textbf{SOL:}}
%:!}
%:\keyDef{use margins}{True/False}{True}{
%: Whether or not to put the part problem labels and the problem headers outside the
%: normal margins.
%:}
%:\keyDef{front page}{file}{}{
%:Path to a front page that will be loaded at every \buildset
%:Where to load the front page. The commands \refCom{setName} is available.
%:}
%:\keyDef{style margins}{yes/no}{yes}{
%:Whether or not to put problem headers and part problem headers out in the margins.
%:}
%:-
%:$packageOptions=
%:$$packageOptions
