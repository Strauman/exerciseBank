%!TEX root = main.tex
\makeatletter
\edef\@isTrue{1}
\edef\@isFalse{0}
%:§config
%:You can do a lot of configurations on this package, and probably
%: even more to come in later versions!
%:=\ShowNumbers
%: Shows the numbers in fron of the part problems and intros that should be used with
%: \exclude and \select
%:-
\gdef\ShowNumbers{\gdef\@displayMetaCounter{\isTrue}}
%:§env
%:=\DisplaySolutions
%:Turns on the solutions, so they are shown.
%:-
\gdef\@DisplaySolutions{\@isFalse}
\gdef\DisplaySolutions{\xdef\@DisplaySolutions{\isTrue}\@latex@warning{Showing solutions}}
\gdef\@SolutionsOnly{\@isFalse}
\gdef\SolutionsOnly{\xdef\@SolutionsOnly{\@isTrue}}
%:§config
%:-
% ---- \exercisebanksetup ----- %
\pgfkeys{
/exbanksetup/.is family, /exbanksetup,
default/.style = {
  part problems = On,
  tighten paragraphs = True,
  problem header={\normalfont\Large\bfseries\@tr{Problem}~\theproblemcounter},
  part problem header={\large\textbf{(\theproblemcounter\alph{partproblemcounter})}},
  solution header={\large{\textbf{\@tr{Solution}:}}},
  exercise directory=.,
  figure root directory=\@exercisesDir,
},
% Expanded macros
exercise directory/.estore in = \@exercisesDir, %<- excepted from opt-syntax
figure root directory/.estore in = \@figrootDir, %<- excepted from opt-syntax
% Unexpanded macros
problem header/.store in = \exbank@opt@problemHeader,
part problem header/.store in = \exbank@opt@partProblemHeader,
solution header/.store in = \exbank@opt@solutionHeader,
% Bools
part problems/.style = {switches/#1/.get = \exbank@opt@partProblems},
tighten paragraphs/.style = {switches/#1/.get = \exbank@opt@tightenparagraphs},
% True/False
switches/.cd,
% True
  On/.initial = \@isTrue,
  on/.initial = \@isTrue,
  T/.initial =  \@isTrue
% False
  Off/.initial = \@isFalse,
  off/.initial = \@isFalse,
  F/.initial = \@isFalse,
}
%:=\exercisebanksetup{[key/values]}
\newcommand\exercisebanksetup[1]{
  \pgfkeys{/exbanksetup, #1}%
}
\exercisebanksetup{default}
\AtBeginDocument{\xdef\tpd{\the\prevdepth}}
\gdef\tighten@paragraph{%
\if\exbank@opt@tightenparagraphs\@isTrue\relax%
  \if\@SolutionsOnly\@isFalse
    \if\@DisplaySolutions\@isFalse
      \filbreak%
    \fi
  \fi
\fi%
}
\gdef\tighten@paragraph@solutions{%
\if\exbank@opt@tightenparagraphs\@isTrue\relax%
  \if\@SolutionsOnly\@isFalse
    \if\@DisplaySolutions\@isTrue
      \filbreak%
    \fi
  \fi
\fi%
}
\gdef\tighten@paragraph@always{%
\if\exbank@opt@tightenparagraphs\@isTrue\relax%
\filbreak%
\fi%
}


%: Here is a list of the different keys and their meaning
%:\keyDef{part problems}{On/Off}{On}{
%: This is whether or not to do part problems. E.g. 1a), 1b) etc.
%: If this is turned Off, then the part problems will be treated as problems
%:}
%:\keyDef{tighten paragraphs}{True/False}{True}{
%: Disabling this will prevent
%: the package from attempting to prevent part problems to scatter across pages
%:}
%:\keyDef{problem header}{macro}{see below}{
%: This sets the problem header. To access the translation of the problem text, use \@tr{Problem}, and
%: the problem counter is accessed with \theproblemcounter.\\
%: Defaults to\\
%: \brackets{\normalfont\Large\bfseries\@tr{Problem}~\theproblemcounter}.
%:}
%:\keyDef{part problem header}{macro}{see below}{
%: This sets the problem header. To access the current problem, use \theproblemcounter, and then the current
%: part problem \thepartproblemcounter. To make it a letter, as per default use \alph{partproblemcounter}\\
%: Default is:\\
%: \large\textbf{(\theproblemcounter\alph{partproblemcounter})}
%:}
%:\keyDef{solution header}{string}{see below}{
%:\large{\textbf{\@tr{Solution}:}}
%:}
%:\keyDef{exercise directory}{dir}{./}{
%:This key is used for setting the default exercise directory.
%:}
%:\keyDef{figure root directory}{dir}{\meta{exercise directory}}{
%:Exercisebank automatically allows you to \input{} and \includegraphics{} from
%:the same folder folder that your exercise is in, as well as a folder with the
%:same name as the exercise file. This is elaborated in the intro section.
%:However, you might want to put the figures inside a different directory.
%:For example if you have one directory containing your problems called \!texttt{exercises/},
%:set with \refKey{exercise directory}, and a file containing the exercise:
%: \!texttt{exercises/faradaysLaw/ACGenerate.tex}.
%:Then by default figures in the directories \!texttt{exercises/faradaysLaw/} and
%:\!texttt{exercises/faradaysLaw/ACGenerate/} can be included by just doing \includegraphics.
%:And if you want to change the root of this directory to be \!texttt{figures}, such that
%:exercisebank looks for figures in \!texttt{figures/faradaysLaw/} and \!texttt{figures/faradaysLaw/ACGenerate/}
%:You would use this option in the setup: \exercisebanksetup{figure root directory=figures}
%:}
%:! \exercisebanksetup{exercise directory=exercises,part problems=Off,solution header={\textbf{SOL:}}}
%:-
