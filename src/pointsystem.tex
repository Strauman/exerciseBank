%!TEX root = main.tex
\makeatletter
\gdef\exb@currentPoints{0}
\gdef\exb@currentPointsStyle{}
%current points style
%:$packageOptions.=\keyDef{current points style}{style}{see below}{
%:$packageOptions.= This is how the points will be shown in the part problem headers.
%:$packageOptions.= the default code is by \!gh{thorstengrote} (see example below)
%:$packageOptions.=}
%:$packageOptions.=\!begin{dispListing}
%:$packageOptions.=~\!ifnum\!exb@currentPoints=\!z@%
%:$packageOptions.=    \!ifnum\!totalpoints=\!z@\!else%
%:$packageOptions.=      \!phantom{0p}%
%:$packageOptions.=    \!fi%
%:$packageOptions.=  \!else%
%:$packageOptions.=    \!exb@currentPoints p%
%:$packageOptions.=  \!fi~\!ignorespaces
%:$packageOptions.=\!end{dispListing}
%:$packageOptions.=\keyDef{disable points}{bool}{false}{
%:$packageOptions.= Whether or not to disable the point system.
%:$packageOptions.=}
%:-
\newif\ifexb@enablepoints\global\exb@enablepointstrue
\exercisebanksetup{%
	current points style/.store in=\exb@currentPointsStyle,
	current points style={\ifnum\exb@currentPoints=\z@\ifnum\totalpoints=\z@\else\phantom{0p}\fi\else\exb@currentPoints p\fi\ignorespaces},
	disable points/.is if={exb@enablepoints}
}
\ifexb@enablepoints
	\AtEndDocument{%
		\immediate\write\@auxout{\string\gdef\string\totalpoints{\the\c@exb@points}}%
	}
	\At\PartProblemHeaderSuffix{\exb@currentPointsStyle}
\fi
\newcounter{exb@points}
\setcounter{exb@points}{0}
\providecommand\totalpoints{}
%:$problemKeys.=\keyDef{points}{number}{0}{
%:$problemKeys.= The number of points the next exercise is worth.
%:$problemKeys.= you can retrieve the total points using \totalpoints
%:$problemKeys.=}
%:-
%:§problemopts
%:=\totalpoints
%:Contains the total number of points for all exercises in the current set
%:-
\nextproblem{points/.store in=\exb@currentPoints}
% \@dlog{\string\exbank@opt@partProblemHeader\meaning\exbank@opt@partProblemHeader}
\At\BeginPartproblem{%
\addtocounter{exb@points}{\exb@currentPoints}%
}
\At\EndPartproblem{\gdef\exb@currentPoints{0}}

%:=\exercisepoints
%: Contains the total number of points in the current exercise (or "Problem")
%:-
\providecommand\exercisepoints{}
% Setting up \exercisepoints (for getting sum of part problem points in exercise)
\newcounter{exb@tmp@exercisepoints}
\setcounter{exb@tmp@exercisepoints}{0}
% First reset counters and check if the aux file has defined these current
% exercisepoints. Points for a given exercise is saved in the auxfile
% under \exb@exercisepoint@\theproblemcounter.
\At\VeryBeginProblem{
	\xdef\exb@expoints@macname{exb@exercisepoints@\the\c@problemcounter}
	\ifcsname \exb@expoints@macname\endcsname%
		\xdef\exercisepoints{\csname\exb@expoints@macname\endcsname}
	\else%
		\gdef\exercisepoints{}
	\fi
	\setcounter{exb@tmp@exercisepoints}{0}
}
\At\BeginPartproblem{%
\addtocounter{exb@tmp@exercisepoints}{\exb@currentPoints}%
}
\At\EndProblem{%
% Now write the total number of points to the .aux file
\immediate\write\@auxout{\string\expandafter\string\gdef\string\csname\space exb@exercisepoints@\the\numexpr\the\c@problemcounter\endcsname{\the\c@exb@tmp@exercisepoints}}%
}
%
% \At\VeryBeginProblem{%
%
% }
