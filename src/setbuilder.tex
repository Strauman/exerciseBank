%!TEX root = main.tex
% This file contains the logic of the set making and set building.
% It also decides whether or not a problem,intro and/or solution
% should be built
%:§makingSets
%:-
\let\ea = \expandafter
\newcounter{metacounter}
\setcounter{metacounter}{0}
\gdef\exb@emptyList{0}
\newif\ifexb@tagbuildmode
\global\exb@tagbuildmodefalse
\global\let\emptyList\exb@emptyList
\global\let\exb@tagList\exb@emptyList
%
\gdef\ifppMode#1{
  \def\mname{\exbank@macroname{#1}}
  \strif\mname\ppMode
}
\gdef\isppMode#1#2{
  \edef\mname{\exbank@macroname{#1}}
  \strif{\exbank@macroname{#1}}{\ppMode}\relax#2\fi
}
% \ppMode is what list mode we're in. Either it's \pm@Exclude or \pm@Select
\gdef\@ifppMode#1{%
\ifnum\pdfstrcmp{\exbank@macroname{#1}}{\ppMode}=\z@%
\expandafter\@firstofone%
\else%
\expandafter\@noneofone%
\fi%
}%
\def\pm@Exclude{exclude}
\def\pm@Select{select}
\def\pm@Normal{normal}

%% Deciding whether problem (and accompanied solution) should be shown
\let\ipm\@ifppMode
\let\T@\isTrue
\let\F@\isFalse
% \ifnum\pdfstrcmp{\exb@tagList}{\exb@emptyList}=\z@\relax
%   \xdef\DisplayProblem{\ipm\exclude\T@\ipm\select\F@}\else%
%   \isin{\exb@currentTags}{\exb@tagList}{}{\xdef\DisplayProblem{\F@}}%
% \fi
\At\DecideProblemDisplay{%
    \ifnum\pdfstrcmp{\ppList}{\emptyList}=\z@\relax%
      \xdef\DisplayProblem{\ipm\exclude\T@\ipm\select\F@}
    \fi
    \exb@int@isin{\themetacounter}{\ppList}{
      \xdef\DisplayProblem{\ipm\exclude\F@\ipm\select\T@\ipm\normal\T@}
    }{
      \xdef\DisplayProblem{\ipm\exclude\T@\ipm\select\F@\ipm\normal\T@}
    }
% \@dlog{DP:\DisplayProblem - \themetacounter}
% \@dlog{mlist:\ppList}
% \@dlog{\exb@currentTags}
% \def\DisplayProblem{0}
\ifexb@tagbuildmode
\ifnum\pdfstrcmp{\DisplayProblem}{\T@}=\z@\relax
%TODO: MakeIntersection
\exb@intersection@any{\exb@currentTags}{\exb@tagList}{\global\exb@opthidesfalse}{\global\exb@opthidestrue}
\fi
\fi
}

%% Makeset arg handler
\pgfkeys{
 /makeset/.is family, /makeset,
 default/.style = {noheadarg=\isFalse, introarg=\isFalse},
 intro/.style = {introarg=\isTrue},
 nohead/.style = {noheadarg=\isTrue},
 introarg/.estore in = \introarg,
 noheadarg/.estore in = \noheadarg
}
% When a \makeset is made, it adds the set name to the
% \@listOfSets for use in \sprite
\gdef\@listOfSets{}
\gdef\@spriteMode{\isFalse}
\gdef\exbank@spriteSets{\emptyList}
%:=\spritesets{setlist}
%: This is a command that is used by sprite to determine what sets are shown in sprite. If this is not given, sprite uses all sets given in \makset
%:-
\long\gdef\spritesets#1{
  \gdef\exbank@spriteSets{#1}
}
%% Generates sets that can be iterated over
%% \makeset[intro|nohead]{\select{path/to/file}{1,2}, \exclude{file}{1,2}, file}
%% Where file is a file in the \setExercisesDir-folder (default it's the same folder you have the file). See  and 1,2 are the part problem numbers
%:=\makeset[intro,nohead]{filable}
%: This command is the one you use to make a set! Later you use \buildset to build the sets you make. The \meta{filable} argument is either the name of the file relative to the \setExercisesDir-path (default is nothing, so it's in the root path), or you could use the \select or \exclude to  respectively cherry pick or exclude exercises. (See their docs).\\
%: \oarg{intro} this counts the intro environment as a part problem, so that you can \select or \exclude the intro\\
%: \oarg{nohead} prevents the builder from adding a problem header. This is handy if you want to create an exercise that is composed of multiple parts. You can use the \phead to insert the problem header where you want it
%:! \makeset[nohead]\{\phead, \select{myexercise}{1,2,3}}
%:-
\global\let\exlist@protect\noexpand
\newcommand\makeset[3][]{
  \xdef\thissetid{#2}
  \ea\xdef\csname setkeys@\thissetid\endcsname{#1}
  \pgfkeys{/makeset, default, #1}%
  % Is intro sent?
  \if\introarg\isTrue%
    \gdef\introarg{\isTrue}
    \ea\gdef\csname setlist@#2@intro\endcsname{\isTrue}
  \fi
  \if\noheadarg\isTrue%
    \ea\gdef\csname setlist@#2@nohead\endcsname{\isTrue}
  \fi
  \ea\gdef\csname setlist@#2\endcsname{\exlist@protect{#3}}
  \def\setmacro{\unexpanded\expandafter{\csname setlist@#2\endcsname}}
  % \g@addto@macro\@listOfSets{\csname setlist@#2\endcsname,}
  \g@addto@macro\@listOfSets{,#2}
}
%:=\about{text}
%:This contains information about an exercise set. It is intended to be on the top of an
%:exercise, explaining short what the exercise is about. It's only visible when using \sprite
%:-
%% It gets defined in \sprite when use
\long\gdef\about#1{}

%:=\sprite[PiP]
%: This is a way to visualize all exercises. It takes one optional argument which is how many pages
%: inside one page. Defaults to 4
%: \!begin{marker}If \sprite is used, it should be the only command in \begin{document}\end{document}\!end{marker}
%:-
\newcommand\sprite[1][4]{
% \exercisebanksetup{part problem header={\Large\textbf{\thepartproblemcounter}}}
\squeeze
\gdef\@spriteMode{\isTrue}
\long\def\about##1{{\Large\textbf{About}:\\[1.1em]##1\\[1.5em]}}
\pgfpagesuselayout{#1 on 1}[a4paper,border shrink=5mm]
\ifnum\pdfstrcmp{\exbank@spriteSets}{\emptyList}=\z@\relax%
  \edef\sprite@setlist{\ea\@secondoftwo\@listOfSets}
  %readlist with star removes whitespaces in the items
  \readlist*\list@sprite@setlist\sprite@setlist
  \foreachitem\set\in\list@sprite@setlist{
    \edef\theset{{\set}}
    \if\theset\empty\relax\else%
    \buildset{\set}%
    \fi%
  }
\else
  \edef\sprite@setlist{\exbank@spriteSets}
  \readlist*\list@sprite@setlist\sprite@setlist
  \foreachitem\set\in\list@sprite@setlist{
    \edef\theset{{\set}}
    \if\theset\empty\relax\else%
    \buildex{\set}%
    \fi%
  }
\fi
}
%:§ref
%:=\exerciseFile
%:This is a `read-only' macro that contains the name of the current exerciseFile
%:-
%:§makingSets
%% Sets the variables \ppList and \exerciseFile based on the current set
\gdef\exbank@setEnv@normal#1{
\exbank@setEnv{#1}{normal}
}
\newcommand{\exbank@setEnv}[3][{-1}]{
  \if\@spriteMode\isFalse%
    \gdef\ppList{#1}
    \gdef\exerciseFile{#2}
    \gdef\ppMode{#3}
  \else
    \gdef\ppList{\emptyList}
    \gdef\exerciseFile{#2}
    \gdef\ppMode{\pm@Normal}
  \fi

}
%% Converts {A}{1,2,3} into [{1,2,3}]{A}{exclude} such that it can be sent
%% as optional arguments to exbank@setEnv. The last is set as ppMode (exclude/select/mix)
%:=\exclude{exerciseFileName}{Comma separated numbers}
%:As you can see in the intro section of the documentation, this is for excluding partproblems
%:To be used in \refCom{makeset}
% :-
\newcommand{\exclude}[2]{[{#2}]{#1}{exclude}}

%:=\select{exerciseFileName}{Comma separated numbers}
%:As you can see in the intro section of the documentation, this is for cherry picking partproblems
%:To be used in \refCom{makeset}
% :-
\newcommand{\select}[2]{[{#2}]{#1}{select}}

%% Build one exercise
\gdef\buildex#1{
  \makeset{#1}{#1}
  \buildset{#1}
}
%% Commands that can be sent to makeset
%% Problem header
\gdef\exbank@def@makeset@command#1#2{%
  \edef\macroname{\exbank@macroname{#1}}%
  \xdef#1{?{@\macroname}}%
  \ea\gdef\csname @\macroname\endcsname{#2}%
}
\let\exbank@isFirstProblem\isTrue
\exbank@def@makeset@command{\phead}{
  \if\exbank@isFirstProblem\isTrue%
    \let\exbank@isFirstProblem\isFalse%
  \else%
    \Trigger\EndProblem%
  \fi%
  \Trigger\BeginProblem%
  \Trigger\VeryBeginProblem%
}
\exbank@def@makeset@command{\pbreak}{\clearpage}
%:-
%:=\DeclareExerciseCommand{command}{actions}
%:This can be used later in makesets. E.g.
%:\DelcareExerciseCommand{\pbreak}\brackets{\clearpage} will make
%:\pbreak behave like \clearpage in the set:
%:!\makeset[nohead]{%
%:! \phead,
%:! myExercise,
%:! \pbreak,
%:! myExerciseOnNewPage%
%:!}
%:-
\let\DeclareExerciseCommand\exbank@def@makeset@command
%:=\exec{macros}
%:You can use this in \makeset to execute commands between problems.
%:E.g. to insert a new page in between two problems when using nohead:
%:!\makeset[nohead]{2}{\phead,my/exercise,\exec{\clearpage},next/exercise}
%:-
%---- Protectors: ----%
%% \ex@protect:
%% \def\hello{\exec{\textbf{world}}} ~> :\noexpand{\textbf{world}}
%% Then when it's tested
%% \let\ex@protect\unexpanded
%% \edef\hello{\hello} ~> :\textbf{world}
\global\let\ex@protect\noexpand
\long\gdef\exec#1{:\ex@protect{#1}}
%%Logic for building a set
\gdef\@countIntros{\isFalse}
\gdef\exb@setpgf#1{%
\edef\keystring{{/makeset,default,#1}}%
\ea\pgfkeys\keystring}

\newcommand\buildset[2][]{%
  \def\oarg{#1}%
  \@ifundefined{setlist@#2@intro}{\gdef\@countIntros{\isFalse}}{\gdef\@countIntros{\isTrue}\@dlog{Counting intros}}%
  \@ifundefined{setlist@#2@nohead}{\gdef\nohead{\isFalse}}{\gdef\nohead{\isTrue}}%
  %% Setting the global setName
%:=\setName
%:This variable prints out the name of your set that you sent to \buildset.
%:The following example prints "Exercise set number 1" and "Exercise set number 2" on the top of each set
%:!\At\StartBuildset{
%:! Exercise set number \setName
%:!}
%:! %... \makesets here ...%
%:!\begin{document}
%:! \buildset{1}{myexercise}
%:! \buildset{2}{myexercise}
%:!\end{document}
%:-
  \xdef\setName{#2}%
  % Set keys for this set in case user set some keys
  \ifcsname setkeys@\setName\endcsname%
  \xdef\skeys{\csname setkeys@#2\endcsname}%
  % \xdef\skeys{default, \skeys}
  \exb@setpgf\skeys%
  \fi%
  %% Warnins if the set doesn't exist
  \@ifundefined{setlist@#2}{%
    \@latex@error{Couldn't find set #2. Did you remember to do \string\makeset{#2}{}? }
    \stop\bye
  }{}
  %% Sets the current setlist from the variable
  %% generated in makeset
  \let\exlist@protect\unexpanded%
  \edef\exbank@setlist{\csname setlist@#2\endcsname}%
  \let\exlist@protect\noexpand%
  \if\@spriteMode\isFalse%
    \if\exb@frontpage\@isFalse\else\input{\exb@frontpage}\fi
    \Trigger\StartBuildset%
  \fi%
  % ---- BEGIN setlist divider ----%
  % The following code distinguish between the different things that can be sent to \makeset.
  % At time of writing that is:
  % \select, \exclude, \exec, or makeset@commands: \pbreak and \phead
  % \select and \exclude expands to [#2]{#1}{select|exclude} i.e. \exclude{file}{1,2,3} -> [1,2,3]{file}{exclude}
  % and are se to exbank@setEnv for parsing. The last expanded "argument" is the \@ppmode
  % or exclude / select eg [1,3]{some,ex}{select}
  % makesetcommands are expanded to ?{@somecommand}, where @somecommand is to be executed.
  % e.g. \phead -> ?{@phead} and the code runs \@phead at appropriate time.
  % For \exec, it expands to :{#1}, and the code runs #1 at appropriate time.
  % e.g. \exec{\textbf{HELLO}} -> :{\textbf{HELLO}}.
  % So if the first character of the current item in the list is a ? or :
  % then execute makeset@command or exec respectively.
  % However, if it's a [, then \select or \exclude is invoked and send this to exbank@setEnv with the args.
  % else it's a normal filename and should be sent as \exbank@setEnv{THEFILENAME}{normal}, where normal
  % tells the rest of the program that all exercises should be included.
  %
  % Is it a makeset@command?
  \gdef\@delegateFileInfo{\@ifnextchar?\@execute@makeset@command\@is@exec}%
  % Is it an exec command?%
  \def\@is@exec{\@ifnextchar:\@execute@exec\@is@file}%
  % Is it a \select/\exclude or is it just a filename?%
  \gdef\@is@file{\@ifnextchar[\exb@setEnv@withOptargs\X@exb@setEnv}%
  \gdef\X@exb@setEnv##1{%
    \def\continueLoop{\isTrue}%
    \@dlog{Processing normal file "\exerciseFileInfo.tex"}%
    \exbank@setEnv@normal{\exerciseFileInfo}\bgroup\nullfont%
  }
  \gdef\exb@setEnv@withOptargs{%
    \def\continueLoop{\isTrue}\ea\exbank@setEnv\exerciseFileInfo\bgroup\nullfont%
  }

  \gdef\@execute@makeset@command ?##1{%
    \edef\inner{\@firstofone##1}%
    \@dlog{Executing macro \@backslashchar\@gobble##1}%
    \csname \inner\endcsname%
    #1\def\continueLoop{\isFalse}\ea\bgroup%
  }
  \def\@execute@exec:{%
    \def\continueLoop{\isFalse}\@dlog{Executing custom command}\bgroup%
  }
%% Loop through list
  \readlist*\setlist\exbank@setlist%
  \foreachitem\exerciseFileInfo\in\setlist{%
    \def\continueLoop{\isTrue}%
    %% ppList is \emptyList (0) if it's not a list
    \gdef\ppList{\emptyList}%
    %% Counter used for iterating over partproblems
    %% (Cant' use partproblemcounter since some of them might be excluded)
    \setcounter{metacounter}{0}%
    %% Checking whether we have optional args
    %% I.e. one of the arguments are sent via \exclude
    %% And then send it to \exbank@setEnv
    %If \exerciseFileInfo starts with a [, then optional arguments
    %that are to be sent to \exbank@setEnv. That is from \exclude or \select
    \let\ex@protect\unexpanded%
    \edef\exerciseFileInfo{\exerciseFileInfo}%
    \ea\@delegateFileInfo\exerciseFileInfo\egroup%
    \Trigger\InputExercise%
    \if\continueLoop\isTrue%
    \let\ex@protect\noexpand%
%% TRIGGER EVERTYHING
  \Trigger\PathControl%
%:$triggers.=\Trigger\InputExercise:\\ Triggers before a file is included\\\
%:$triggers.=\Trigger\BeginProblem:\\ Triggers before a file is included, but only if problem headers are to be written (no [nohead] given)\\
%:$triggers.=\Trigger\EndProblem:\\ Triggers right after problem is included if [nohead] \emph{not} given\\
%:$triggers.=\Trigger\BeginBuildset:\\ Triggers right before a set has begun building (not if \dac{sprite} is used). You might want to put your set-header here\\
%:$triggers.=\Trigger\EndBuildset:\\ Triggers when a set has stopped building (not if \dac{sprite} is used)\\
    \if\nohead\isFalse%
      \if\@spriteMode\isFalse%
        \Trigger\BeginProblem%
      \fi%
    \fi%
    %% \VeryBeginProblem is only for stuff that
    %% is intended to be a part of the problem
    %% e.g. problem header is using this.
    \if\nohead\isFalse\Trigger\VeryBeginProblem\fi%
    \if\@spriteMode\isTrue\textbf{\exerciseFile.tex\\}\fi%
    %% \incl = \ input because otherwise latexpanded would try.
    %% to input the file
    \IfFileExists{\@exercisesDir/\exerciseFile}{%
      \incl{\@exercisesDir/\exerciseFile}%
      \if\@spriteMode\isTrue%
        \setcounter{partproblemcounter}{0}%
      \fi
    }{
      \@latex@error{Could not find \@exercisesDir/\exerciseFile. Maybe it is because the default exercise directory is now changed to the same directory that your main file is in. To set default exercise directory to exercises, do \string\exercisebanksetup{exercise directory=exercises}}{}
      \stop\bye
    }
    %% More triggers
    \if\nohead\isFalse\Trigger\EndProblem\fi%
    \fi%<- End \continueLoop
  }
  \if\@spriteMode\isFalse%
    \Trigger\EndBuildset%
  \fi
  \setcounter{problemcounter}{0}%
  \setcounter{partproblemcounter}{0}%
  \clearpage%
}
%:=\buildsets{list}
%:This command will generate multiple sets:
%:!\buildsets{set,set2,set3}
%:-
\newcommand\buildsets[1]{%
	\xdef\csvlist{#1}%
	\readlist*\items\csvlist%
	\foreachitem\set\in\items{%
		\buildset{\set}%
	}
}
%:=\buildtags{tagslist}{setslist}
%:This command will generate multiple sets:
%:!\buildsets{tag,tag2,tag3}{set,set2,set3}
%:-
\newcommand\buildtags[2]{%
  \global\exb@tagbuildmodetrue
  \xdef\csvtags{#1}%
  \xdef\csvlist{#2}
	\xdef\csvsets{#2}%
	% \readlist*\tagsItems\csvtags%
  \readlist*\setsItems\csvsets%
	\foreachitem\set\in\setsItems{%
    \xdef\exb@tagList{\csvtags}
		\buildset{\set}%
	}
  \global\exb@tagbuildmodefalse
}
