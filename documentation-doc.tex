The documentation is ultimately formatted by tcolorbox. But it is parsed from comments in the code by a custom script.
The syntax for doc parsing is explained through these examples:

It's programmed to end when the next comment is none of the four styles, but to be safe, use:
%:-
What section are we in?

%:§sectionVar=Section name
%:-

Macro definition
%:=\someMacro[oarg]{marg1}{marg2}
%: Description
%: of
%: someMacro goes here
%:-

Environment definition
%:=\\begin{someEnv}[oarg]{marg1}{marg2}
%: Description
%: of
%: environment
%:-

Example
%:! Exampleusage of \someMacro
%:! or any \othermacro
%:! goes here
%:-

Example within definition
%:=\someMacroWithExample
%:! My example
%:! Within the def
%: Description of anything else
%: Goes like this
%:! another example within def
%:-

You can make variables:
%:$MYVAR.=Something
%:$MYVAR.=\Hello\World
%:$MYVAR.= \somewhere{overtherainbow}
%:$MYVAR.= \!foo{bar}

Then later you can use it in two ways:
The first way is to just print what the variable has now:
%:MYVAR IS
%:$MYVAR

The second way is to wait until the entire script has finished and then
print the variables:
%:$$MYVAR

Macro escaping:
Most macros will be escaped automatically with \dac. That is:

%:Talking about some \macro
will print in the documentation as Talking about some \dac{macro}.
(\dac is defined by \let\dac\DocumentAuxCommand, which displays the macro nicely in tcolorbox)

To avoid escaping macros, then you'd use \!macro. E.g.
%: it is \!emph{very} important to note that.

Contents of variables and descriptions will always be macro-escaped.
